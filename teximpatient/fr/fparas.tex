% This is part of the book TeX for the Impatient.
% Copyright (C) 2003 Paul W. Abrahams, Kathryn A. Hargreaves, Karl Berry.
% Copyright (C) 2004 Marc Chaudemanche pour la traduction fran�aise.
% See file fdl.tex for copying conditions.

\input fmacros
\chapter {Commandes \linebreak pour composer \linebreak des paragraphes}

\chapterdef{paras}

Cette section couvre les commandes qui traitent des caract\`eres, des mots, des 
lignes et des paragraphes entiers. Pour une explication des conventions 
utilis\'ees dans cette section, voir les \headcit{Descriptions de 
commandes}{cmddesc}.

\begindescriptions

\section {Caract\`eres et accents}

%==========================================================================
\subsection {Lettres et ligatures pour alphabets Europ\'eens}

\begindesc
\xrdef{fornlets}
\bix^^{ligatures}
^^{symboles sp\'eciaux}
^^{alphabets europ\'eens}
%
\ctsx AA {Lettre scandinave \AA}
\ctsx aa {Lettre scandinave \aa}
\ctsx AE {Ligature \AE}
\ctsx ae {Ligature \ae}
\ctsx L  {Lettre Polonaise \L}
\ctsx l  {Lettre Polonaise \l}
\ctsx O  {Lettre Danoise/Norv\'egienne \O}
\ctsx o  {Lettre Danoise/Norv\'egienne \o}
\ctsx OE {Ligature \OE}
\ctsx oe {Ligature \oe}
\ctsx ss {Lettre Allemande \ss}
\explain
Ces commandes produisent diverses lettres et ligatures des alphabets 
europ\'eens. Elles sont utiles pour des mots et expressions occasionnels 
dans ces langues---mais si vous devez composer une grande quantit\'e de 
texte dans une langue europ\'eenne, vous feriez probablement mieux 
d'utiliser une version de \TeX\ adapt\'ee \`a cette langue.\footnote{Le \TeX\ 
Users Group (\xref{ressources}) peut vous fournir des informations sur les 
versions europ\'eennes de \TeX.}

Vous aurez besoin d'un espace apr\`es ces commandes quand vous les 
utiliserez dans un mot pour que \TeX\ traite les lettres suivantes comme 
\'el\'ement du mot et non en tant qu'\'el\'ement de la commande. Vous 
n'avez pas besoin d'\^etre en \minref{mode math\'ematique} pour utiliser 
ces commandes.

\example
{\it les \oe uvres de Moli\`ere}
|
\produces
{\it les \oe uvres de Moli\`ere}
\endexample
\eix^^{ligatures}
\enddesc

%==========================================================================
\subsection {Symboles sp\'eciaux}

\begindesc
^^{caract\`eres sp\'eciaux}
%
\easy\ctspecialx # \ctsxrdef{@pound} {signe di\`ese \#}
\ctspecialx $ \ctsxrdef{@bucks} {signe dollar \$}
\ctspecialx % \ctsxrdef{@percent} {signe pourcentage \%}
\ctspecialx & \ctsxrdef{@and} {esperluette \&}
\ctspecialx _ \ctsxrdef{@underscore} {soulign\'e \_}
\ctsx lq {quote gauche \lq}
\ctsx rq {quote droite \rq}
\aux\ctsx lbrack crochet gauche [
\aux\ctsx rbrack crochet droit ]
\ctsx dag {symbole dague \dag}
\ctsx ddag {symbole double dague \ddag}
\ctsx copyright {symbole copyright \copyright}
\ctsx P {symbole paragraphe \P}
\ctsx S {symbole section \S}
\explain

Ces commandes produisent diverses marques et caract\`eres sp\'eciaux. Les 
cinq premi\`eres commandes sont n\'ecessaires parce que \TeX\ attache par 
d\'efaut des significations sp\'eciales aux caract\`eres (|#|, |$|, |%|, 
|&|, |_|). vous n'avez pas besoin d'\^etre en \minref{mode math\'ematique} 
pour employer ces commandes.

Vous pouvez utiliser le signe dollar de la police Computer Modern italique 
pour obtenir le symbole ^{livre sterling}, comme montr\'e dans l'exem\-ple 
ci-dessous.

\example
\dag It'll only cost you \$9.98 over here, but in England
it's {\it \$}24.98.
|
\produces
\dag It'll only cost you \$9.98 over here, but in England
it's {\it \$}24.98.
\endexample
\enddesc

\begindesc
\cts TeX {}
\explain
Cette commande produit le logo \TeX. souvenez vous de la faire suivre par 
|\!vs| ou de l'inclure dans un \minref{groupe} quand vous voulez un espace 
apr\`es lui.

\example
A book about \TeX\ is in your hands.
|
\produces
A book about \TeX\ is in your hands.
\endexample
\enddesc

\begindesc
\cts dots {}
\explain
^^{points}
Cette commande produit des ^{points de suspension}, c'est-\`a-dire, trois 
points, dans du texte ordinaire. Elle est pr\'evue pour \^etre utilis\'ee 
en \'ecriture math\'ematique, pour des points de suspension entre des mots 
ordinaires, vous devez utiliser |$\ldot$| \ctsref{\ldots} \`a la place. 
Puisque |\dots| inclut son propre espace, vous ne devez pas le faire suivre 
de |\!vs|. 
\example
The sequence $x_1$, $x_2$, \dots, $x_\infty$
does not terminate.
|
\produces
The sequence $x_1$, $x_2$, \dots, $x_\infty$
does not terminate.
\endexample
\enddesc

\see ``Divers symboles math\'ematiques ordinaires'' (\xref{specsyms}).
%==========================================================================
\subsection {Caract\`eres Arbitraires}

\begindesc
\bix^^{caract\`eres}
\cts char {\<charcode>}
\explain
Cette commande produit le caract\`ere situ\'e \`a la position \<charcode> 
de la police courante. 
\example
{\char65} {\char `A} {\char `\A}
|
\produces
{\char65} {\char `A} {\char `\A}
\endexample
\enddesc

\begindesc
\cts mathchar {\<mathcode>}
\explain
Cette commande produit le caract\`ere math\'ematique dont la classe, la 
famille et la position dans la police sont donn\'ees par \<mathcode>. Elle 
n'est l\'egale qu'en mode math\'ematique. 
\example
\def\digger{\mathchar "027F} % Like \spadesuit in plain TeX.
% Class 0, family 2, font position "7F.
$\digger$
|
\produces
\def\digger{\mathchar "027F}
% class 0, family 2, font position "7F
$\digger$
\endexample
\enddesc

\see |\delimiter| (\xref\delimiter).
\eix^^{caract\`eres}

%==========================================================================
\subsection {Accents}

\begindesc
^^{accents}
\xrdef{accents}
%
\ctspecialx ' \ctsxrdef{@prime} {^{accent aigu} comme dans \'e}
\ctspecialx . \ctsxrdef{@dot} {^{accent point} comme dans \.n}
\ctspecialx = \ctsxrdef{@equal} {^{accent macron} comme dans \=r}
\ctspecialx ^ \ctsxrdef{@hat} {^{accent circonflexe} comme dans  \^a}
\ctspecialx ` \ctsxrdef{@lquote} {^{accent grave} comme dans \`e}
\ctspecialx " \ctsxrdef{@quote} {^{accent tr\'ema} comme dans \"o}
\ctspecialx ~ \ctsxrdef{@not} {^{accent tilde} comme dans \~a}
\ctsx c {^{accent c\'edille} comme dans \c c}
\ctsx d {^{accent point en dessous} comme dans \d r}
\ctsx H {^{accent tr\'ema Hongrois} comme dans \H o}
\ctsx t {^{accent tie-after} comme dans \t uu}
\ctsx u {^{accent bref} comme dans \u r}
\ctsx v {^{accent  circonflexe invers\'e} comme dans \v o}
\explain
Ces commandes produisent des accents en texte ordinaire. Vous devez 
normalement laisser un espace apr\`es ceux not\'es par une simple lettre. 
(see ``Espaces'', \xref{espaces}).

\example
Add a soup\c con of \'elan to my pin\~a colada.
|
\produces
Add a soup\c con of \'elan to my pin\~a colada.
\endexample

\margin{`See also' moved to end of group, replacing the one there.}
\enddesc

\begindesc
\cts i {}
\cts j {}
\explain
Ces commandes produisent des versions sans point des lettres `i' et `j'. 
Vous devez les utiliser au lieu des `i' et `j' ordinaires quand vous mettez 
un accent sur ces lettres dans du texte ordinaire. 
^^{lettres sans points}
Utilisez les commandes ^|\imath| et ^|\jmath| (\xref\imath) pour des `i' et 
`j' sans point dans des formules math\'ematiques.
\example
long `i' as in l\=\i fe  \quad \v\j
|
\produces
long `i' as in l\=\i fe  \quad \v\j
\endexample
\enddesc

\begindesc
\cts accent {\<charcode>}
\explain
^^{accents}
Cette commande met un accent au-dessus du caract\`ere suivant cette 
commande. L'accent est le caract\`ere de position \<charcode> dans la 
police courante. \TeX\ suppose que l'accent a \'et\'e con\c cu pour 
s'adapter sur un caract\`ere d'une hauteur de $1$\thinspace ex dans la 
m\^eme police que l'accent. Si le caract\`ere \`a accentuer est plus haut 
ou plus petit, \TeX\ ajuste la position en cons\'equence. Vous pouvez 
changer de \minref{police} entre l'accent et le caract\`ere suivant, cela 
dessine le caract\`ere accent et le caract\`ere \`a accentuer venant de 
diff\'erentes polices. Si le caract\`ere accent n'est pas vraiment pr\'evu 
pour \^etre un accent, \TeX\ ne se plaindra pas~; il composera juste quelque 
chose de ridicule. 
\example
l'H\accent94 otel des Invalides
% Position 94 of font cmr10 has a circumflex accent.
|
\produces
l'H\accent94 otel des Invalides
% Position 94 of font cmr10 has a circumflex accent.
\endexample
\see Accents math\'ematiques (\xref{mathaccent}).
\enddesc
%==========================================================================
\subsection {Ligatures aux limites}

\begindesc
\bix^^{ligatures}
\cts noboundary {}
\explain
Vous pouvez d\'efaire une ligature ou un cr\'enage que \TeX\ applique au 
premier ou au dernier caract\`ere d'un mot en mettant |\noboundary| juste 
avant ou apr\`es ce mot. Certaines polices pr\'evues pour des langues 
autres que l'anglais contiennent un caract\`ere sp\'ecial de fronti\`ere 
que \TeX\ met au commencement et \`a la fin de chaque mot. Le caract\`ere 
de fronti\`ere n'occupe aucun espace et est invisible une fois imprim\'e. 
Il permet \`a \TeX\ de fournir un traitement typographique diff\'erent aux 
caract\`eres de d\'ebut ou de fin de mot, puisque le caract\`ere de 
fronti\`ere peut faire partie d'une s\'equence de caract\`eres devant 
\^etre cr\'enel\'ee ou remplac\'ee par une ligature (aucune police 
standard de \TeX\ ne contient ce caract\`ere de fronti\`ere).  L'effet de 
|\noboundary| est de supprimer le caract\`ere de fronti\`ere s'il est l\`a, 
et de ce fait d'emp\^echer \TeX\ d'identifier la ligature ou le cr\'enage.
\eix^^{ligatures}
\enddesc

%==========================================================================
\section {S\'electionner des polices}

\xrdef{selfont}

%==========================================================================
\subsection {Polices particuli\`eres}

\begindesc
^^{polices}
%
\ctsx fivebf {utilise une police grasse de $5$ points}
\ctsx fivei {utilise une police math\'ematique italique de $5$ points}
\ctsx fiverm {utilise une police romaine de $5$ points}
\ctsx fivesy {utilise une police de symbole math\'ematique de $5$ points}
\ctsx sevenbf {utilise une police grasse de $7$ points}
\ctsx seveni {utilise une police math\'ematique italique de $7$ points}
\ctsx sevenrm {utilise une police romaine de $7$ points}
\ctsx sevensy {utilise une police de symbole math\'ematique de $7$ points}
\ctsx tenbf {utilise une police grasse de texte de $10$ points}
\ctsx tenex {utilise une police d'extension math\'ematique de $10$ points}
\ctsx teni {utilise une police math\'ematique italique de $10$ points}
\ctsx tenrm {utilise une police romaine de texte de $10$ points}
\ctsx tensl {utilise une police romaine pench\'ee de $10$ points}
\ctsx tensy {utilise une police de symbole math\'ematique de $10$ points}
\ctsx tenit {utilise une police italique de $10$ points}
\ctsx tentt {utilise une police de type machine \`a \'ecrire de $10$ points}
\explain
Le texte suivant ces commandes est compos\'e par \TeX\ dans la police 
indiqu\'ee. Normalement vous devriez enfermer une de ces commandes de choix 
de police dans un groupe avec le texte devant \^etre compos\'e dans la 
police choisie. En dehors d'un groupe une commande de choix de police est 
efficace jusqu'\`a la fin du document (\`a moins que vous ne l'annuliez 
avec une autre commande de ce genre). 
\example
See how I've reduced my weight---from
120 lbs.\ to {\sevenrm 140 lbs}.
|
\produces
See how I've reduced my weight---from
120 lbs.\ to {\sevenrm 140 lbs}.
\endexample
\enddesc

\begindesc
\cts nullfont {}
\explain
Cette commande s\'electionne une police, construite dans \TeX, qui ne 
comporte aucun caract\`ere. \TeX\ l'utilise en remplacement d'une police 
non d\'efinie dans une famille des polices math\'ematiques.
\enddesc

%==========================================================================
\subsection {Styles de caract\`ere}

\xrdef{seltype}
\begindesc
^^{styles de caract\`ere}
\easy\ctsx bf {utilise un style gras}
\ctsx it {utilise un style italique}
\ctsx rm {utilise un style romain}
\ctsx sl {utilise un style pench\'e}
\ctsx tt {utilise un style machine \`a \'ecrire}
\explain
Ces commandes choisissent un style de caract\`ere sans changer l'oeil ou 
la taille du caract\`ere.\footnote{\TeX\ ne fournit pas de commandes 
pr\'ed\'efinies pour ne changer que la taille, par exemple, |\eightpoint|. 
La fourniture de telles commandes exigerait un grand nombre de 
polices, lesquelles ne seraient jamais utilis\'ees. De telles commandes 
ont \'et\'e cependant employ\'ees dans la composition de \texbook.} 
Normalement vous enfermeriez une de ces commandes de s\'election de 
style de caract\`ere dans un groupe, avec le texte devant \^etre compos\'e 
dans la police choisie. En dehors d'un groupe une commande de 
s\'election de style est efficace jusqu'\`a la fin du document (\`a moins 
que vous l'effaciez avec une autre commande de ce genre). 
\example
The Dormouse was {\it not} amused.
|
\produces
The Dormouse was {\it not} amused.
\endexample
\enddesc

\see ``Polices dans des formules math\'ematiques'' (\xref{mathfonts}).
%==========================================================================
\section {Majuscule et minuscule}

\begindesc
\bix^^{conversion de casse}
\bix^^{majuscules//conversion en}
\bix^^{minuscules//conversion en}
\cts lccode {\<charcode> \tblentry{nombre}}
\cts uccode {\<charcode> \tblentry{nombre}}
\explain
Les valeurs |\lccode| et |\uccode| pour les $256$ caract\`eres possibles 
indiquent la correspondance entre les formes minuscules et majuscules 
des lettres. Ces valeurs sont utilis\'ees par les commandes |\lowercase| 
et |\uppercase| respectivement et par l'algorithme de c\'esure de \TeX.

\TeX\ initialise les valeurs de |\lccode| et |\uccode| comme suit~:

\ulist\compact
\li Le |\lccode| d'une lettre minuscule est le code {\ascii} de cette lettre.
\li Le |\lccode| d'une lettre majuscule est le code {\ascii} de la lettre minuscule
correspondante.
\li Le |\uccode| d'une lettre majuscule est le code {\ascii} de cette lettre.
\li Le |\uccode| d'une lettre minuscule est le code {\ascii} de la lettre majuscule
correspondante.
\li Les |\lccode| et |\uccode| d'un caract\`ere qui n'est pas une lettre sont \`a z\'ero.
\endulist

La majeure partie du temps, il n'y a aucune raison de changer ces valeurs, 
mais vous pourriez devoir les changer si vous vous servez d'une langue qui 
a plus de lettres que l'anglais.
\example
\char\uccode`s \char\lccode`a \char\lccode`M
|
\produces
\char\uccode`s \char\lccode`a \char\lccode`M
\endexample
\enddesc

\begindesc
\cts lowercase {\rqbraces{\<liste de token>}}
\cts uppercase {\rqbraces{\<liste de token>}}
\explain ^^{conversion de casse}
Ces commandes convertissent les lettres de la \<token list>, c'est-\`a-dire, 
les tokens avec le code de cat\'egorie $11$, dans leurs formes minuscules 
et majuscules. La conversion d'une lettre est d\'efinie par sa valeur de 
table |\lccode| (pour une minuscule) ou |\uccode| (pour une majuscule). Les 
tokens de la liste qui ne sont pas des lettres ne sont pas affect\'ees---m\^eme 
si les tokens sont des appels de \minref{macro} ou d'autres commandes qui 
se d\'eveloppent en lettres.
\example
\def\x{Cd} \lowercase{Ab\x} \uppercase{Ab\x}
|
\produces
\def\x{Cd} \lowercase{Ab\x} \uppercase{Ab\x}

\eix^^{conversion de casse}
\eix^^{majuscules//conversion en}
\eix^^{minuscules//conversion en}
\endexample
\enddesc

%==========================================================================
\section {Espace inter-mot}

\begindesc
\bix^^{espace//inter-mots}
\easy\ctsbasic {\\\vs}{}
\blankidxref\ctsxrdef{@space}
\explain
Cette commande produit explicitement un espace entre les mots appel\'e 
``^{espace contr\^ol\'e}''. un espace contr\^ol\'e est utile quand une lettre 
appara\^\i t juste apr\`es une commande, ou dans n'importe quelle autre 
circonstance o\`u vous ne voulez pas que deux tokens soient reli\'es dans 
la sortie. La place produite par |\!vs| est ind\'ependante de la 
ponctuation pr\'ec\'edente, c'est-\`a-dire, que son facteur d'espace 
(\xref\spacefactor) est $1000$. 

Par ailleurs, si vous voulez imprimer le caract\`ere `\vs' ^^{espace 
visible} que nous avons utilis\'e ici pour noter un espace, vous pouvez 
l'obtenir en saisissant |{\tt \char `\ }|. 

\example
The Dormouse was a \TeX\ expert, but he never let on.
|
\produces
The Dormouse was a \TeX\ expert, but he never let on.
\endexample
\enddesc

\begindesc
\cts space {}
\explain
Cette commande est \'equivalente \`a la saisie d'un caract\`ere espace. Elle 
diff\`ere de ^|\ | du fait que sa largeur \emph{peut} \^etre affect\'e par 
la ponctuation pr\'ec\'edente. 
\example
Yes.\space No.\space Maybe.\par
Yes.\!vs!.No.\!vs!.Maybe.

|
\produces
Yes.\space No.\space Maybe.\par
Yes.\ No.\ Maybe.
\endexample
\enddesc

\begindesc
\ctsact  ^^M \xrdef{@newline}
\explain
Cette construction produit le caract\`ere ^{fin de ligne}. Elle a 
normalement deux effets quand \TeX\ la rencontre dans votre source~: 
\olist
\li Elle agit en tant que commande, en introduisant un espace (si elle 
appara\^\i t \`a l'extr\'emit\'e d'une ligne non vide) ou un 
token |\par| (si elle appara\^\i t \`a l'extr\'emit\'e d'une ligne vide). 
^^|\par//d'une ligne vide| 
\li Elle termine la ligne du source, faisant que \TeX\ ignore les 
caract\`eres restants sur la ligne.
\endolist
\noindent
Cependant, |^^M| \emph{ne} fait \emph{pas} terminer la ligne quand il 
appara\^\i t dans le contexte |`\^^M|, d\'enotant le code ASCII du control-M 
(le nombre $13$). Vous pouvez changer la signification du |^^M| en lui 
donnant un \minref{code de cat\'egorie} diff\'erent. Voir 
\xrefpg{twocarets} pour une explication plus g\'en\'erale de la notation 
|^^|.
\example
Hello.^^MGoodbye.
Goodbye again.\par
The \char `\^^M\ character.\par
% The fl ligature is at position 13 of font cmr10
\number `\^^M\ is the end of line code.\par
Again, \number `^^M is the end of line code,
isn't it? % 32 is the ASCII code for a space
|
\produces
{\catcode `\^ = 7 % disable indexing use within this display
Hello.^^MGoodbye
Goodbye again.\par
The \char `\^^M\ character.\par
\number `\^^M\ is the end of line code.\par
Again, \number `^^M is the end of line code,
isn't it?}
\endexample
\enddesc

\begindesc
\easy\ctsact ~ \xrdef{@not}
\explain
Le caract\`ere \minref{actif} `|~|' appel\'e un ``^{tilde}'', produit 
une espace normale entre deux mots et lie ces mots pour 
qu'une coupure de ligne ne se produise pas entre eux. Vous devez 
employer un tilde dans n'importe quel contexte o\`u une coupure de 
ligne serait embrouillante, par exemple, avant une initiale moyenne, 
apr\`es une abr\'eviation telle que ``Dr'' ou apr\`es ``Fig.'' dans ``Fig.~8''. 

\example
P.D.Q.~Bach (1807--1742), the youngest and most
imitative son of Johann~S. Bach, composed the
{\sl Concerto for Horn and Hardart}.
|
\produces
\margin{The inversion of dates is deliberate---cf. Peter Schickele.}
P.D.Q.~Bach (1807--1742), the youngest and most
imitative son of Johann~S. Bach, composed the
{\sl Concerto for Horn and Hardart}.
\endexample\enddesc

\begindesc
\easy\ctspecial / \ctsxrdef{@slash}
\explain
Chaque caract\`ere d'une \minref{fonte} \TeX\ a une correction d'``^{italique}''
li\'ee \`a lui, bien que la correction d'italique soit 
normalement \`a z\'ero pour un caract\`ere d'une police non pench\'ee (montante). 
La correction d'italique indique l'espace suppl\'ementaire qui 
est n\'ecessaire quand vous passez d'une police inclin\'ee (pas 
n\'ecessairement une police italique) \`a une police non pench\'ee. 
L'espace suppl\'ementaire est n\'ecessaire parce qu'un caract\`ere inclin\'e 
d\'epasse dans l'espace qui le suit, r\'eduisant l'espace quand le caract\`ere 
suivant est non pench\'e. Le dossier de 
m\'etriques d'une police inclut la correction d'italique de chaque 
caract\`ere dans la police. 
^^{fichiers de m\'etriques//correction italique dans}

La commande |\/|
produit une correction d'^{italique} pour le caract\`ere pr\'ec\'edent. Vous 
devez ins\'erer une correction d'italique quand vous passez d'une 
police inclin\'ee vers une police droite, sauf quand le caract\`ere suivant 
est un point ou une virgule. 
\example
However, {\it somebody} ate {\it something}: that's clear.

However, {\it somebody\/} ate {\it something\/}:
that's clear.
|
\produces
However, {\it somebody} ate {\it something}: that's clear.

However, {\it somebody\/} ate {\it something\/}:
that's clear.
\endexample
\enddesc

\begindesc
\cts frenchspacing {}
\cts nonfrenchspacing {}
\explain
^^{espacement inter-mots}
\TeX\ ajuste normalement l'espacement entre les mots en 
accord avec les signes de ponctuation. Par exemple, il ins\`ere de l'espace 
suppl\'ementaire \`a la fin d'une phrase et ajoute un certain \'etirement 
au \minref{ressort} apr\`es n'importe quel signe de ponctuation. La 
commande |\frenchspacing| indique \`a \TeX\ de rendre l'espacement 
entre les mots ind\'ependant de la ponctuation, alors que la commande  
|\nonfrenchspacing| indique \`a \TeX\ d'utiliser les r\`egles normales 
d'espacement. Si vous n'indiquez pas |\frenchspacing|, vous 
obtiendrez l'espacement normal de \TeX. 

Voyez \xrefpg{periodspacing} pour des conseils sur la fa\c con de 
contr\^oler le traitement de la ponctuation de \TeX\ \`a la fin des 
phrases. 

\example
{\frenchspacing  An example: two sentences. Right? No.\par}
{An example: two sentences. Right? No. \par}%
|
\produces
{\frenchspacing  An example: two sentences. Right? No.\par}
{An example: two sentences. Right? No. \par}%
\endexample

\enddesc

\begindesc
\cts obeyspaces {}
\explain
\TeX\ condense normalement une suite de plusieurs espaces en un 
espace simple. |\obeyspaces| demande \`a \TeX\ de produire un 
espace dans le r\'esultat pour chaque espace dans le source. 
Cependant,|\obeyspaces| ne met pas d'espace au d\'ebut d'une ligne~; 
pour cela nous vous recommandons la commande de 
|\obey!-white!-space| d\'efinie dans |eplain.tex|
(\xref{ewhitesp}).
^^|\obeywhitespace|
|\obeyspaces| est souvent utile quand vous composez quelque chose, 
un source informatique par exemple, dans une police \`a espacement fixe (une police dans 
laquelle chaque caract\`ere a le m\^eme espacement) et que vous voulez montrer 
exactement \`a quoi ressemble chaque ligne du source. 

Vous pouvez utiliser la commande |\obeylines| (\xref{\obeylines}) 
pour obtenir que \TeX\ suive les bords de ligne de votre source. 
|\obeylines| est souvent utilis\'e en combinaison avec |\obeyspaces|. 

\example 
These     spaces    are    closed   up
{\obeyspaces but   these  are     not   }.
|
\produces
These     spaces    are    closed   up
{\obeyspaces but   these  are     not   }.
\endexample
\enddesc

\begindesc
\cts spacefactor {\param{nombre}}
\cts spaceskip {\param{ressort}}
\cts xspaceskip {\param{ressort}}
\cts sfcode {\<charcode> \tblentry{nombre}}
\explain
Ces \minref{param\`etres} primitifs affectent combien d'espace \TeX\ met 
entre deux mots adjacents, c'est-\`a-dire, l'^{espacement inter-mots}. 
L'espacement normal entre les mots est assur\'e par la police courante. 
Pendant que \TeX\ traite une \minref{liste horizontale}, il garde trace 
du facteur d'^{espacement} $f$ dans |\spacefactor|. pendant qu'il 
traite chaque caract\`ere du source $c$, il met \`a jour $f$ selon la 
valeur de $f_c$, le code de facteur d'espacement de $c$ (voir ci-
dessous). Pour la plupart des caract\`eres, $f_c$ est \`a $1000$ et \TeX\ 
met $f$ \`a $1000$. (la valeur initiale de $f$ est \'egalement $1000$.) 
Quand \TeX\ voit un espace entre les mots, il ajuste la taille de cet 
espace en multipliant l'\'etirement et le r\'etr\'ecissement de cet espace 
par $f/1000$ et $1000/f$ respectivement. d'o\`u~:
\olist\compact
\li Si $f=1000$, l'espace inter-mots garde sa valeur normale.
\li If $f<1000$, l'espace inter-mots prend moins d'\minref{\'etirement}
et plus de \minref{r\'etr\'ecissement}.
\li If $f>1000$, l'espace inter-mots prend plus d'\minref{\'etirement}
et moins de \minref{r\'etr\'ecissement}.
\endolist
En outre, si $f\ge2000$ l'espace inter-mots est encore augment\'e par le 
param\`etre ``d'espace suppl\'ementaire'' li\'e \`a la police courante.

Chaque caract\`ere saisi $c$ a une adresse dans la table de |\sfcode| 
(code de facteur d'espacement). L'adresse de la table |\sfcode| est 
ind\'ependante de la police. Normalement \TeX\ met simplement $f$ \`a 
$f_c$ apr\`es avoir trait\'e $c$. De toute mani\`ere~:
\ulist
\li Si $f_c$ est \`a z\'ero, \TeX\ laisse $f$ sans changement. Ainsi, un 
caract\`ere tel que `|)|' dans \plainTeX, pour lequel $f_c$ est \`a z\'ero, 
est essentiellement transparent pour le calcul de l'espacement entre 
les mots.
\li Si $f<1000<f_c$, \TeX\ met $f$ \`a $1000$ plut\^ot que $f_c$, c'est-\`a-
dire, qu'il refuse d'augmenter $f$ tr\`es rapidement.
\endulist
La valeur de |\sfcode| pour un point est normalement de $3000$, c'est 
pourquoi \TeX\ met habituellement un espace suppl\'ementaire apr\`es un 
point 
% > to \ge here, too, as above.
(voir la r\`egle ci-dessus pour le cas $f\ge2000$). Les caract\`eres non 
lettre d'une liste horizontale, par exemple, les traits verticaux, 
agissent g\'en\'eralement comme des caract\`eres avec un facteur d'espace de 
$1000$. 

Vous pouvez changer le facteur d'espace explicitement en donnant une 
valeur num\'erique diff\'erente \`a |\spacefactor|. Vous pouvez \'egalement 
augmenter l'espacement normal entre les mots en donnant une valeur 
num\'erique diff\'erente \`a |\xspaceskip| ou \`a |\spaceskip|~:
\ulist
\li |\xspaceskip| sp\'ecifie le ressort utilis\'e quand $f\ge2000$~;
dans le cas o\`u
|\xspaceskip| est \`a z\'ero, les r\`egles normales s'appliquent.
\li |\spaceskip| sp\'ecifie le ressort utilis\'e quand $f<2000$ ou bien 
quand \hbox{|\xspaceskip|} est \`a z\'ero~; si |\spaceskip| est \`a z\'ero, les 
r\`egles normales s'appliquent. L'\'etirement et le r\'etr\'ecissement du 
ressort |\spaceskip|, comme le ressort ordinaire entre les mots, est 
modifi\'e selon la valeur de $f$. 
\endulist
Voir \knuth{page~76}{88--89} pour les r\`egles pr\'ecises que \TeX\ utilise pour 
calculer le \minref{ressort} entre les mots et \knuth{pages~285--287}{331--334} 
pour les ajustements faits \`a |\spacefactor| apr\`es diff\'erents items dans 
une liste horizontale. 
\eix^^{espace//inter-mots}
\enddesc

%==========================================================================
\section {Centrer et justifier les lignes}

\begindesc
\bix^^{centrage}
\bix^^{cadr\'e \`a gauche}
\bix^^{cadr\'e \`a droite}
\bix^^{justification}
\easy\cts centerline {\<argument>}
\cts leftline {\<argument>}
\cts rightline {\<argument>}
\explain
La commande |\centerline| produit un \minref{hbox} aussi large que 
la ligne courante et place \<argument> au centre de la bo\^\i te. Les 
commandes |\leftline| et |\rightline| sont analogues~; elles placent 
\<argument> au bord gauche ou droite de la bo\^\i te. Si vous voulez 
appliquer une de ces commandes \`a plusieurs lignes cons\'ecutives, vous 
devez l'appliquer \`a chacune individuellement. Voir 
\xrefpg{eplaincenter} pour une autre approche.

N'utilisez pas ces commandes dans un paragraphe---si vous le faites, 
\TeX\ ne pourra probablement pas couper le paragraphe en lignes et se 
plaindra au sujet d'un overfull hbox.
\example
\centerline{Grand Central Station}
\leftline{left of Karl Marx}
\rightline{right of Genghis Khan}
|
\produces
\centerline{Grand Central Station}
\leftline{left of Karl Marx}
\rightline{right of Genghis Khan}

\eix^^{centrage}
\eix^^{cadr\'e \`a gauche}
\eix^^{cadr\'e \`a droite}
\eix^^{justification}

\endexample
\enddesc

\begindesc
\easy\cts line {\<argument>}
\explain
Cette commande produit une \minref{hbox} contenant l'\<argument>. Cette 
hbox est exactement aussi large que la ligne courante, c'est-\`a-dire, 
qu'elle s'\'e\-tend de la marge droite \`a la gauche.
\example
\line{ugly \hfil suburban \hfil sprawl}
% Without \hfil you'd get an `underfull box' from this.
|
\produces
\line{ugly \hfil suburban \hfil sprawl}%
\endexample

\enddesc

\begindesc
^^{texte recouvrant}
\cts llap {\<argument>}
\cts rlap {\<argument>}
\explain
Ces commandes vous permettent de produire un texte qui recouvre celui 
qui est \`a gauche ou \`a droite de la position actuelle. |\llap| recule 
par la largeur de l'\<argument> et puis compose l'\<argument>. Le 
|\rlap| est semblable, sauf qu'il compose l'\<argument> d'abord et 
recule. |\llap| et |\rlap| sont utiles pour placer du texte en dehors 
des marges courantes. |\llap| et |\rlap| effectuent leur travail en 
cr\'eant une \minref{box} de largeur z\'ero.

Vous pouvez \'egalement utiliser |\llap| ou |\rlap| pour construire des 
caract\`eres sp\'eciaux par ^{surimpression}, mais ne l'essayez pas \`a 
moins d'\^etre s\^ur que les caract\`eres que vous utilisez aient la m\^eme 
largeur (ce qui est le cas pour une police \`a espacement fixe telle que 
^|cmtt10|, la Computer Modern $10$-points ^{police de type machine \`a 
\'ecrire})..
^^{police Computer Modern}
\example
\noindent\llap{off left }\line{\vrule $\Leftarrow$
left margin of examples\hfil right margin of examples
$\Rightarrow$\vrule}\rlap{ off right}
|
\produces
\noindent\llap{off left }\line{\vrule $\Leftarrow$
left margin of examples\hfil right margin of examples
$\Rightarrow$\vrule}\rlap{ off right}
\endexample

%\example
%{\tt O\llap{!|}}
%|
%\produces
%{\cm \tt O\llap{\char `|}}
%\endexample

\nobreak % don't lose the \see
\enddesc

\see |\hsize| (\xref{\hsize}).

%==========================================================================
\section {Formation des paragraphes}

\subsection {d\'ebuter, finir et indenter des paragraphes}

\begindesc
\bix^^{paragraphes//formation}
\ctspecial par \ctsxrdef{@par}
\explain
Cette commande termine un paragraphe et met \TeX\ dans le mode 
\minref{vertical}, pr\^et \`a ajouter plus d'articles sur la page. 
Puisque \TeX\ convertit une ligne blanche dans votre fichier source 
en un \minref{token} |\par|, vous n'avez normalement pas besoin de 
saisir |\par| explicitement pour finir un paragraphe. 

Un point important est que |\par| n'indique pas \`a \TeX\ de commencer 
un paragraphe~; il ne lui dit que de le finir. \TeX\ d\'ebute un 
paragraphe quand il est en mode vertical ordinaire (tel qu'il est 
apr\`es un |\par|) et rencontre un article en soi horizontal tel qu'une 
lettre. En tant qu'\'el\'ement du c\'er\'emonial de d\'ebut de paragraphe, \TeX\ 
^^{paragraphes//d\'ebuter des} 
ins\`ere une quantit\'e d'espace vertical donn\'ee par le param\`etre 
|\parskip|  (\xref{\parskip }) et indente le paragraphe par un espace 
horizontal donn\'e par |\parindent| (\xref{\parindent }). 

Vous pouvez normalement effacer tout l'espace inter-paragraphe 
produit avec |\par| en mettant la commande |\vskip -\lastskip|. Ceci 
peut \^etre utile quand vous \'ecrivez une \minref{macro} cens\'e travailler 
de la m\^eme mani\`ere si elle est pr\'ec\'ed\'ee ou non par une ligne blanche. 

Vous pouvez obtenir que \TeX\ prenne certaines mesures sp\'eciales au 
d\'ebut de chaque paragraphe en pla\c cant des instructions dans 
^|\everypar| (\xref\everypar). 

Voir \knuth{pages~283 et 286}{328 et 332} pour les effets pr\'ecis de |\par|.

\example
\parindent = 2em
``Can you row?'' the Sheep asked, handing Alice a pair of
knitting-needles as she was speaking.\par ``Yes, a little%
---but not on land---and not with needles---'' Alice was
starting to say, when suddenly the needles turned into oars.
|
\produces
\parindent = 2em
``Can you row?'' the Sheep asked, handing Alice a pair of
knitting-needles as she was speaking.\par ``Yes, a little%
---but not on land---and not with needles---'' Alice was
starting to say, when suddenly the needles turned into oars.
\endexample
\enddesc

\begindesc
\cts endgraf {}
\explain
Cette commande est un synonyme de la commande primitive 
^|\par|. Elle est utile quand vous avez red\'efini ^|\par| mais voulez 
toujours avoir acc\`es \`a la d\'efinition originale de |\par|. 
\enddesc

\begindesc
\cts parfillskip {\param{ressort}}
\explain
^^{paragraphes//ressort \`a la fin des}
Ce param\`etre d\'esigne le ressort horizontal que \TeX\ insert \`a la fin 
d'un paragraphe. La valeur par d\'efaut de |\parfillskip| est |0pt plus 
1fil|, qui fait que la derni\`ere ligne d'un paragraphe soit compl\'et\'ee 
de l'espace vide. Une valeur de |0pt| force \TeX\ \`a finir la derni\`ere 
ligne d'un paragraphe sur la marge gauche. 
\enddesc

\bix^^{indentation}
\begindesc
\easy\cts indent {}
\explain
Si \TeX\ est en mode vertical, comme il l'est apr\`es avoir fini un 
paragraphe, cette commande ins\`ert le ressort inter-paragraphe 
^|\parskip|, met \TeX\ en mode horizontal, d\'ebute un paragraphe et 
indente ce paragraphe de |\parindent|. Si \TeX\ est d\'ej\`a en mode 
horizontal, cette commande produit simplement  un espace blanc de 
largeur |\parindent|. Deux |\indent| \`a la suite produisent deux 
indentations.
^^{indentation}

Comme dans l'exemple ci-dessous, un |\indent| \`a un endroit o\`u \TeX\ 
commencerait de toute fa\c con un paragraphe est superflu. Quand \TeX\ 
est en mode vertical et voit une lettre ou une autre commande en soi 
horizontale, il commence un paragraphe en commutant vers le mode 
horizontal, fait un |\indent| et traite la commande horizontale. 

\example
\parindent = 2em  This is the first in a series of three 
paragraphs that show how you can control indentation. Note
that it has the same indentation as the next paragraph.\par
\indent This is the second in a series of three paragraphs.
It has \indent an embedded indentation.\par
\indent\indent This doubly indented paragraph
is the third in the series.
|
\produces
\parindent = 2em  This is the first in a series of three 
paragraphs that show how you can control indentation. Note
that it has the same indentation as the next paragraph.\par
\indent This is the second in a series of three paragraphs.
It has \indent an embedded indentation.\par
\indent\indent This doubly indented paragraph
is the third in the series.
\endexample
\enddesc


\begindesc
\easy\cts noindent {}
\explain
Si le \TeX\ est en mode vertical,  comme il l'est apr\`es la fin d'un 
paragraphe, cette commande ins\`ert le ressort inter-paragraphe 
^|\parskip|, met \TeX\ dans le mode horizontal, et commence 
un paragraphe non indent\'e. Elle n'a aucun effet dans le mode horizontal, 
c'est-\`a-dire, dans un paragraphe. Commencer un paragraphe avec 
|\noindent| d\'ecommande ainsi l'indentation de |\parindent| qui se 
produirait normalement.

^^{indentation}

Une utilisation commune de |\noindent| est d'interdire l'indentation 
de la premi\`ere ligne d'un paragraphe quand le paragraphe suit un 
certain mat\'eriel affich\'e.

\example
\parindent = 1em
Tied round the neck of the bottle was a label with the
words \smallskip \centerline{EAT ME}\smallskip
\noindent beautifully printed on it in large letters.
|
\produces
\parindent = 1em
Tied round the neck of the bottle was a label with the
words \smallskip \centerline{EAT ME}\smallskip
\noindent beautifully printed on it in large letters.
\endexample
\enddesc

\margin{{\tt\\textindent} moved here from later in the section.}
\begindesc
\cts textindent {\<argument>}

\explain
^^{indentation}
Cette commande indique \`a \TeX\ de commencer un paragraphe et de l'indenter 
de |\par!-indent|, comme d'habitude. \TeX\ alors justifie \`a droite 
l'\<argument> de l'indentation et le fait suivre d'un espace demi-cadratin 
(moiti\'e d'un cadratin). \PlainTeX\ utilise cette commande pour composer les 
notes de pied de page. (\xref\footnote) ^^{notes de pied de page//utiliser 
\b\tt\\textindent\e\ avec} et les articles dans les listes (voir |\item|, 
\xref\item). 

\example
\parindent = 20pt \textindent{\raise 1pt\hbox{$\bullet$}}%
You are allowed to use bullets in \TeX\ even if
you don't join the militia, and many peace-loving
typographers do so.
|
\produces
\parindent = 20pt \textindent{\raise 1pt\hbox{$\bullet$}}%
You are allowed to use bullets in \TeX\ even if
you don't join the militia, and many peace-loving
typographers do so.
\endexample\enddesc

\begindesc
\cts parindent {\param{dimension}}

\explain
Ce \minref{param\`etre} indique la quantit\'e par laquelle la premi\`ere 
ligne de chaque paragraphe doit \^etre indent\'ee. ^^{indentation} Comme 
montr\'e dans l'exemple ci-dessous, placer |\parindent| et ^|\parskip| \`a 
z\'ero est une mauvaise id\'ee puisque alors les coupures de paragraphe ne 
sont plus \'evidentes. 
\example
\parindent = 2em This paragraph is indented by 2 ems.
\par \parindent=0pt This paragraph is not indented at all.
\par Since we haven't reset the paragraph indentation,
this paragraph isn't indented either.
|
\produces
\parindent = 2em This paragraph is indented by 2 ems.
\par \parindent=0pt This paragraph is not indented at all.
\par Since we haven't reset the paragraph indentation,
this paragraph isn't indented either.
\endexample\enddesc

\begindesc
\cts everypar {\param{liste de token}}

\explain
\TeX\ ex\'ecute les commandes des \<token list> toutes les fois qu'il entre dans le mode 
horizontal, par exemple, quand il commence un paragraphe. Par d\'efaut 
|\everypar| est vide, mais vous pouvez prendre des mesures 
suppl\'ementaires au d\'ebut de chaque paragraphe en mettant des commandes 
pour ces actions dans un token list
%
% This \vglue makes the example overwrite the example, but since we are
% not reprinting this page, it doesn't matter.  For reasons I did not
% attempt to track down, a page break happened before the example,
% unlike in the first printing.
% 
%\secondprinting{\vglue-48pt}
et assigner ce token list \`a |\everypar|.
\example
\everypar = {$\Longrightarrow$\enspace}
Now pay attention!!\par
I said, ``Pay attention!!''.\par
I'll say it again!! Pay attention!!
|
\produces
\everypar = {$\Longrightarrow$\enspace}
Now pay attention!\par
I said, ``Pay attention!''.\par
I'll say it again! Pay attention!
\endexample
\enddesc
%\secondprinting{\vfill\eject}

%==========================================================================
\subsection {Formation de paragraphes entiers}

\begindesc
\margin{Cette commande \'etait aussi d\'ecrite dans le chapitre `Pages'. La
description actuelle combine les deux anciennes descriptions.}
\bix^^{coupures de ligne//et formation de paragraphe}
\easy\cts hsize {\param{dimension}}

\explain
Ce \minref{param\`etre} indique la longueur courante de la ^{ligne}, c'est-\`a-dire, 
la largeur habituelle des lignes dans un paragraphe commen\c cant \`a la 
marge gauche. Beaucoup de commandes \TeX emploient 
implicitement la valeur de |\hsize|, comme par exemple, 
|\center!-line| (\xref{\centerline}) et |\hrule| (\xref{\hrule}). En changeant le |\hsize| dans un 
groupe vous pouvez changer la largeur des constructions produites par de 
telles commandes. 

Si vous placez |\hsize| dans un \minref{vbox} qui contient du texte, le 
vbox aura la largeur que vous aurez donn\'ee au |\hsize|. 
^^{vbox//largeur d\'etermin\'ee par \b\tt\\hsize\e}

\PlainTeX\ fixe |\hsize| \`a |6.5in|.

\example
{\hsize = 3.5in % Set this paragraph 3.5 inches wide.
The hedgehog was engaged in a fight with another hedgehog,
which seemed to Alice an excellent opportunity for
croqueting one of them with the other.\par}%
|
\produces
{\hsize = 3.5in 
The hedgehog was engaged in a fight with another hedgehog,
which seemed to Alice an excellent opportunity for croqueting
one of them with the other.\par}%

\doruler{\8\8\8\tick\1\tick\2\tick\1\tick\3}{3.5}{in}
\nextexample
\leftline{\raggedright\vtop{\hsize = 1.5in
Here is some text that we put into a paragraph that is
an inch and a half wide.}\qquad
\vtop{\hsize = 1.5in Here is some more text that
we put into another paragraph that is an inch and a
half wide.}}
|
\produces
\leftline{\raggedright\vtop{\hsize = 1.5in
Here is some text that we put into a paragraph that is
an inch and a half wide.}\qquad
\vtop{\hsize = 1.5in Here is some more text that
we put into another paragraph that is an inch and a
half wide.}}
\endexample
\enddesc

\begindesc
\easy\cts narrower {}

\explain
^^{paragraphes//\'etroit}
Cette commande donne des paragraphes plus \'etroits, en augmentant les marges 
gauches et droites de |\parindent|, l'^{indentation} de paragraphe 
courante. Elle r\'ealise ceci en augmentant |\leftskip| et |\rightskip| de 
|\parindent|. Normalement vous placez |\narrower| au d\'ebut d'un 
\minref{groupe} contenant les paragraphes que vous voulez rendre plus 
\'etroits. Si vous oubliez d'enfermer |\narrower| dans un groupe, vous 
constaterez que tout le reste de votre document aura des paragraphes 
\'etroits.

|\narrower| n'affecte que les paragraphes qui terminent apr\`es que vous 
l'ayez appel\'e. Si vous terminez un groupe |\narrower| avant d'avoir fini un 
paragraphe, \TeX\ ne rendra pas ce paragraphe plus \'etroit.

\example
{\parindent = 12pt \narrower\narrower\narrower
This is a short paragraph. Its margins are indented
three times as much as they would be
had we used just one ``narrower'' command.\par}
|
\produces
{\parindent = 12pt \narrower\narrower\narrower
This is a short paragraph. Its margins are indented
three times as much as they would be
had we used just one ``narrower'' command.\par}
\endexample\enddesc

\begindesc
\cts leftskip {\param{ressort}}
\cts rightskip {\param{ressort}}

\explain
Ces param\`etres indiquent \`a \TeX\ combien de ressort placer aux extr\'emit\'es 
gauches et droites de chaque ligne du paragraphe courant. Nous n'expli\-%
querons que le fonctionnement de |\leftskip| puisque |\rightskip| est 
similaire.

^^{indentation} Vous pouvez augmenter la marge gauche en pla\c cant 
|\leftskip| \`a une \minref{dimension} diff\'erente de z\'ero. Si vous donnez \`a 
|\leftskip| un certain \'etirement, vous pouvez produire du texte ^{cadr\'e \`a 
droite}, c'est-\`a-dire, du texte qui a une marge gauche in\'egale.

Normalement, vous devriez enfermer tout \minref{assignement} \`a |\leftskip| 
dans un \minref{groupe} avec le texte affect\'e afin d'emp\^echer son effet de 
continuer jusqu'\`a la fin de votre document. Cependant, il est injustifi\'e de 
changer la valeur de |\leftskip| \`a l'int\'erieur d'un groupe qui est, lui 
aussi, contenu dans un paragraphe---la valeur de |\leftskip| \`a la 
\emph{fin} d'un paragraphe est celle qui d\'etermine comment \TeX\ coupera le 
paragraphe en lignes. \minrefs{line break} 

\example
{\leftskip = 1in The White Rabbit trotted slowly back
again, looking anxiously about as it went, as if it had
lost something.  {\leftskip = 10in % has no effect
It muttered to itself, ``The Duchess!! The Duchess!! She'll
get me executed as sure as ferrets are ferrets!!''}\par}%
|
\produces
{\leftskip = 1in The White Rabbit trotted slowly back
again, looking anxiously about as it went, as if it had
lost something.  {\leftskip = 10in % has no effect
It muttered to itself, ``The Duchess! The Duchess!
She'll get me executed as sure as ferrets are ferrets!''}\par}%
\nextexample
\pretolerance = 10000 % Don't hyphenate.
\rightskip = .5in plus 2em
The White Rabbit trotted slowly back again, looking
anxiously about as it went, as if it had lost something.
It muttered to itself, ``The Duchess!! The Duchess!! She'll
get me executed as sure as ferrets are ferrets!!''
|
\produces
\pretolerance = 10000 % Don't hyphenate.
\rightskip = .5in plus 2em
The White Rabbit trotted slowly back again, looking
anxiously about as it went, as if it had lost something.
It muttered to itself, ``The Duchess! The Duchess! She'll
get me executed as sure as ferrets are ferrets!''
\endexample
\enddesc

\begindesc
\easy\cts raggedright {}
\cts ttraggedright {}

\explain
Ces commandes font que \TeX\ compose votre document 
``^{cadr\'e \`a gauche}''.  Les espaces inter-mots ont tous 
leur taille naturelle, c'est-\`a-dire, ils ont tous la m\^eme largeur et
ne s'\'etirent ni se r\'etr\'ecissent.
En cons\'equence, leur marge droite n'est g\'en\'eralement pas la m\^eme.
L'alternative, qui est le comportement par d\'efaut de \TeX, est de composer 
votre document justifi\'e,
^^{justification}
c'est-\`a-dire, avec des marges gauches et droites uniformes.
Dans du texte justifi\'e, les espaces inter-mots sont \'etir\'es pour
aligner des marges de droite.
Certains typographes pr\'ef\`erent cadrer \`a gauche parce que
cela \'evite des ``rivi\`eres'' distractives d'espace sur la page imprim\'ee.
\minrefs{justified text}

Vous devez utiliser la commande |\ttraggedright| pour composer du texte dans 
une police non proportionnelle et la commande |\raggedright| command pour 
composer de texte dans toutes les autres polices. 

La plupart du temps vous voudrez appliquer ces commandes \`a un document entier,
mais vous pouvez limiter leurs effets en les englobant
dans un \minref{groupe}.
\example
\raggedright ``You couldn't have it if you {\it did\/}
want it,'' the Queen said. ``The rule is, jam tomorrow
and jam yesterday---but never jam {\it today\/}.''
``It {\it must\/} come sometimes to `jam today,%
thinspace'' Alice objected. ``No, it can't'', said the
Queen. ``It's jam every {\it other\/} day: today isn't
any {\it other\/} day.''
|
\produces
\raggedright ``You couldn't have it if you {\it did\/}
want it,'' the Queen said. ``The rule is, jam tomorrow
and jam yesterday---but never jam {\it today\/}.''
``It {\it must\/} come sometimes to `jam today,%
'\thinspace'' Alice objected. ``No, it can't'', said the
Queen. ``It's jam every {\it other\/} day: today isn't
any {\it other\/} day.''
\endexample
\enddesc

\begindesc
\cts hang {}
\explain
Cette commande indente la seconde ligne et les suivantes d'un paragraphe
de |\parindent|, l'^{indentation} du paragraphe.
(\xref{\parindent}).
Puisque la premi\`ere ligne est d\'ej\`a indent\'ee par |\parindent|
(\`a moins que vous ayez supprim\'e l'indentation avec |\noindent|), le
paragraphe entier appara\^\i t \^etre indent\'e de |\parindent|.

\example
\parindent=24pt \hang  ``I said you {\it looked} like an
egg, Sir,'' Alice gently explained to Humpty Dumpty. ``And
some eggs are very pretty, you know,'' she added.
|
\produces
\parindent=24pt \hang  ``I said you {\it looked} like an
egg, Sir,'' Alice gently explained to Humpty Dumpty. ``And
some eggs are very pretty, you know,'' she added.
\endexample
\enddesc

\begindesc
\cts hangafter {\param{nombre}}
\cts hangindent {\param{dimension}}

\explain
Ces deux \minref{param\`etre}s associ\'es
sp\'ecifient l'``^{indentation r\'emanente}'' pour un paragraphe.
L'indentation r\'emanente indique \`a \TeX\ que certaines lignes
du paragraphe doivent
\^etre indent\'ees et que les lignes restantes doivent avoir leur largeur normale
^^{indentation}
|\hangafter| d\'etermine quelles lignes
sont indent\'ees, tandis que |\hangindent| d\'etermine le montant d'inden\-tation
et si elle appara\^\i t sur la gauche ou sur la droite~: 

\ulist
\li Soit $n$ la valeur de |\hangafter|.  Si $n < 0$, 
Les $-n$ premi\`eres lignes du paragraphe seront indent\'ees.
Si $n\ge0$, toutes sauf les $n$ premi\`eres lignes du paragraphe seront
indent\'ees.

\li Soit $x$ la valeur de |\hangindent|.
Si $x\ge0$, les lignes seront indent\'ees
de $x$ \`a gauche. Si $x<0$, les lignes seront indent\'ees de $-x$ sur
la droite.  
\endulist

Quand vous sp\'ecifiez de l'indentation r\'emanente, elle ne s'applique 
qu'au paragraphe suivant (si vous \^etes en mode vertical) ou au
paragraphe courant (si vous \^etes en mode horizontal).
\TeX\ utilise les valeurs de |\hangafter| et |\hangindent| \`a la fin d'un
paragraphe, quand il coupe ce paragraphe en lignes.\minrefs{line
break}
 
\`A la diff\'erence de la plupart des autres param\`etres de mise en forme de paragraphe,
|\hangafter| et |\hangindent| sont r\'einitialis\'es
au d\'ebut de chaque paragraphe, soit,
$1$ pour |\hangafter| et $0$ pour |\hangindent|.
Si vous voulez composer une suite de paragraphes avec de l'indentation r\'emanente, 
utilisez |\everypar| (\xref{\everypar}).
^^|\everypar//pour indentation r\'emanente|
Si vous sp\'ecifiez |\hangafter| et |\hangindent| de m\^eme que ^|\parshape|,
\TeX\ ignorera |\hangafter| et |\hangindent|.

\example
\hangindent=6pc \hangafter=-2
This is an example of a paragraph with hanging indentation. 
In this case, the first two lines are indented on the left,
but after that we return to unindented text.
|
\produces
\hangindent=6pc \hangafter=-2
This is an example of a paragraph with hanging indentation. 
In this case, the first two lines are indented on the left,
but after that we return to unindented text.
\nextexample
\hangindent=-6pc \hangafter=1
This is another example of a paragraph with hanging
indentation.  Here, all lines after the first have been
indented on the right. The first line, on the other
hand, has been left unindented.
|
\produces
\hangindent=-6pc \hangafter=1
This is another example of a paragraph with hanging
indentation.  Here, all lines after the first have been
indented on the right. The first line, on the other
hand, has been left unindented.
\endexample
\enddesc

\margin{{\tt\\textindent} has been moved to earlier in this section.}

\begindesc
\cts parshape {$n\; i_1 l_1\; i_2 l_2\; \ldots \;i_n l_n$}
\explain
Cette commande sp\'ecifie le gabarit des $n$ premi\`eres lignes d'un para\-graphe.
Le prochain si vous \^etes en mode vertical et le paragraphe courant si vous 
\^etes en mode horizontal.
Les $i$ et les $l$ sont toutes des dimensions. La premi\`ere ligne est indent\'ee de
$i_1$ et a une longueur de $l_1$, la seconde ligne est indent\'ee de $i_2$ et a
une longueur de $l_2$, et ainsi de suite.
si le paragraphe a plus de $n$ lignes, la derni\`ere paire indentation\slash 
longueur est utilis\'ee pour les autres lignes.
Pour parachever des effets sp\'eciaux comme celui montr\'e ici, vous devrez
normalement faire beaucoup d'essais, ins\'erer des cr\'enages ici et l\`a et
choisir votre mot pour remplir le gabarit.

|\parshape|, comme ^|\hangafter| et ^|\hangindent|, n'est effectif que pour
un paragraphe.
Si vous sp\'ecifiez |\hangafter| et |\hangindent| de m\^eme que |\par!-shape|,
\TeX\ ignorera ^|\hangafter| et ^|\hangindent|.
\ifodd\pageno\vfill\eject\fi % so the wineglass is on a single page.

\example
% A small font and close interline spacing make this work
\smallskip\font\sixrm=cmr6 \sixrm \baselineskip=7pt
\fontdimen3\font = 1.8pt \fontdimen4\font = 0.9pt
\noindent \hfuzz 0.1pt
\parshape 30 0pt 120pt 1pt 118pt 2pt 116pt 4pt 112pt 6pt
108pt 9pt 102pt 12pt 96pt 15pt 90pt 19pt 84pt 23pt 77pt
27pt 68pt 30.5pt 60pt 35pt 52pt 39pt 45pt 43pt 36pt 48pt
27pt 51.5pt 21pt 53pt 16.75pt 53pt 16.75pt 53pt 16.75pt 53pt
16.75pt 53pt 16.75pt 53pt 16.75pt 53pt 16.75pt 53pt 16.75pt
53pt 14.6pt 48pt 24pt 45pt 30.67pt 36.5pt 51pt 23pt 76.3pt
Les vins de France et de Californie semblent \^etre 
les plus connus, mais ne sont pas les seuls bons vins. 
Les vins espagnols sont souvent sous-estim\'es, 
et des assez vieux peu\-vent \^etre disponibles \`a des 
prix raisonnables. Pour les vins espagnols, le mill\'esime 
n'est pas important, mais le climat de la r\'egion de 
Bordeaux varie selon les ann\'ees. Seuls certains sont 
bons. Ceux que 
vo\kern -.1pt us de\kern -.1pt v\kern -.1pt ez 
noter tr\`es tr\`es bons~: 
1962, 1964, 1966. 1958, 1959, 1960, 1961, 1964, 1966 sont de bons crus 
californiens. \`A boire avec mod\'eration~! 
|
%\margin{Wineglass text replaced because of permissions problem.}
\produces
% A small font and close interline spacing make this work
\smallskip\font\sixrm=cmr6 \sixrm \baselineskip=7pt
\fontdimen3\font = 1.8pt \fontdimen4\font = 0.9pt
\noindent \hfuzz 0.1pt
\parshape 30 0pt 120pt 1pt 118pt 2pt 116pt 4pt 112pt 6pt 108pt 9pt 102pt
12pt 96pt 15pt 90pt 19pt 84pt 23pt 77pt 27pt 68pt 30.5pt 60pt 35pt 52pt
39pt 45pt 43pt 36pt 48pt 27pt 51.5pt 21pt 53pt 16.75pt 53pt 16.75pt
53pt 16.75pt 53pt 16.75pt 53pt 16.75pt 53pt 16.75pt 53pt 16.75pt
53pt 16.75pt 53pt 14.6pt 48pt 24pt 45pt 30.67pt 36.5pt 51pt 23pt 76.3pt
Les vins de France et de Californie semblent \^etre 
les plus connus, mais ne sont pas les seuls bons vins. 
Les vins espagnols sont souvent sous-estim\'es, 
et des assez vieux peu\-vent \^etre disponibles \`a des 
prix raisonnables. Pour les vins espagnols, le mill\'esime 
n'est pas important, mais le climat de la r\'egion de 
Bordeaux varie selon les ann\'ees. Seuls certains sont 
bons. Ceux que 
vo\kern -.1pt us de\kern -.1pt v\kern -.1pt ez 
noter tr\`es tr\`es bons~:  
1962, 1964, 1966. 1958, 1959, 1960, 1961, 1964, 1966 sont de bons crus 
californiens. \`A boire avec mod\'eration~! \endexample
\eix^^{indentation}
\enddesc

\begindesc
\cts prevgraf {\param{nombre}}
\explain
En mode horizontal, ce param\`etre sp\'ecifie le 
nombre de lignes dans le paragraphe en cours~; en mode vertical,
il sp\'ecifie le nombre de lignes dans le paragraphe pr\'ec\'edent.
\TeX\ ne prend en compte |\prevgraf| qu'apr\`es avoir fini de couper du texte
en ligne, c'est-\`a-dire, sur un affichage math\'ematique ou \`a la fin d'un 
paragraphe.
Voir la \knuth{page~103}{121} pour plus de d\'etails \`a son propos.
\enddesc

\begindesc
\cts vadjust {\rqbraces{\<mat\'eriel en mode vertical>}}
\explain
Cette commande ins\`ere le \<mat\'eriel en mode vertical> sp\'ecifi\'e juste apr\`es 
la ligne de sortie contenant la position o\`u la commande se trouve.
^^{listes verticales//ins\'erer dans des paragraphes}
Vous pouvez l'utiliser, par exemple, pour provoquer un saut de page ou 
ins\'erer de l'espace suppl\'ementaire apr\`es une certaine ligne.

\example
Some of these words are \vadjust{\kern8pt\hrule} to be
found above the line and others are to be found below it.
|
\produces
Some of these words are \vadjust{\kern8pt
\hbox to \hsize{\hfil\vbox{\advance\hsize by -\parindent
\hrule width \hsize}}}
to be found above the line and others are to be found below it.
\endexample
\enddesc

\see |\parindent| (\xref\parindent),
|\parskip| (\xref\parskip) ainsi que |\every!-par| (\xref\everypar).
\eix^^{coupures de ligne//et formation de paragraphe}
\eix^^{paragraphes//formation}

%==========================================================================
\section {coupures de lignes}

%==========================================================================
\subsection {Encourager ou d\'ecourager les coupures de ligne}

\begindesc
\bix^^{coupures de ligne}
\bix^^{coupures de ligne//encouragement ou d\'ecouragement}
\ctspecial break {} \xrdef{hbreak}
\explain
Cette commande force une coupure de ligne,
\`a moins que vous fassiez quelque chose pour remplir la ligne, 
Vous recevrez vraisemblablement une plainte ``underfull hbox''.
|\break| peut aussi \^etre utilis\'e en mode vertical.
\example
Fill out this line\hfil\break and start another one.\par
% Use \hfil here to fill out the line.
This line is underfull---we ended it\break prematurely.
% This line causes an `underfull hbox' complaint.
|
\produces
\hbadness = 10000 % avoid hbadness message
Fill out this line\hfil\break and start another one.\par
% Use \hfil here to fill out the line.
This line is underfull---we ended it\break prematurely.
% This line causes an `underfull hbox' complaint.
\endexample\enddesc

\begindesc
\ctspecial nobreak {} \xrdef{hnobreak}
\explain
Cette commande emp\^eche une coupure de ligne l\`a o\`u elle
devrait se faire normalement.
|\nobreak| peut aussi \^etre utilis\'ee en mode vertical.
\example
Sometimes you'll encounter a situation where
a certain space\nobreak\qquad must not get lost.
|
\produces
Sometimes you'll encounter a situation where
a certain space\nobreak\qquad must not get lost.
\endexample
\enddesc

\begindesc
\ctspecial allowbreak {} \xrdef{hallowbreak}
\explain
Cette commande demande \`a \TeX\ d'autoriser une coupure de ligne l\`a o\`u elle
n'arriverait pas normalement.
Elle est plus souvent utile dans une formule math\'ematique, car \TeX\ est
r\'eticent \`a couper de telles lignes. ^^{coupures de ligne//dans des formules math\'ematiques}
|\allowbreak| peut aussi \^etre utilis\'ee en mode vertical.
\example
Under most circumstances we can state with some confidence
that $2+2\allowbreak=4$, but skeptics may disagree.
\par For such moronic automata, it is not difficult to
analyze the input/\allowbreak output behavior in the limit.
|
\produces
Under most circumstances we can state with some confidence
that $2+2\allowbreak=4$, but skeptics may disagree.
\par For such moronic automata, it is not difficult to
analyze the input/\allowbreak output behavior in the limit.
\endexample\enddesc

\begindesc
\ctspecial penalty {\<nombre>} \xrdef{hpenalty}
\explain
Cette commande produit un \'el\'ement de \minref{p\'enalit\'e}.
L'\'el\'ement de p\'enalit\'e rend \TeX\ plus ou moins d\'esireux de couper une ligne
\`a l'endroit o\`u cet \'el\'ement est plac\'e.
Une p\'enalit\'e n\'egative, c'est-\`a-dire, un bonus, encourage une coupure de ligne~;
une p\'enalit\'e positive d\'ecourage une coupure de ligne.  
Une p\'enalit\'e de $10000$ ou plus emp\^eche une coupure quelle qu'elle soit,
tandis qu'une p\'enalit\'e de $-10000$ ou moins force une coupure.
|\penalty| peut aussi \^etre utilis\'ee en mode vertical.
%\secondprinting{\vfill\eject}
\example
\def\break{\penalty -10000 } % as in plain TeX
\def\nobreak{\penalty 10000 } % as in plain TeX
\def\allowbreak{\penalty 0 } % as in plain TeX
|
\endexample
\enddesc

{\vglue-\baselineskip\vskip0pt}
\begindesc
\cts obeylines {}
\explain
\TeX\ normalement traite une fin de ligne comme un espace.
|\obeylines| demande \`a \TeX\ de traiter chaque fin de ligne comme
une fin de paragraphe, for\c cant ainsi une coupure de ligne.
|\obeylines| est souvent utile quand vous composez un po\`eme ou
un programme informatique.
^^{vers, composer des}^^{po\'esie, composer de la}^^{programmes informatiques, composer des}
Si quelques lignes sont plus longues que la longueur de ligne r\'eelle
(|\hsize|\tminus|\parindent|),
sinon,
vous aurez une coupure de ligne suppl\'ementaire dans ces lignes.

Parce que \TeX\ ins\`ere le ressort |\parskip| (\xref\parskip)
entre des lignes contr\^ol\'ees par |\obeylines| (puisqu'il pense que chaque
ligne est un paragraphe), vous devrez normalement mettre |\parskip| \`a z\'ero 
quand vous utiliserez |\obeylines|.

Vous pouvez utiliser la commande ^|\obeyspaces| (\xref{\obeyspaces}) pour que
\TeX\ prenne en compte tous les espaces d'une ligne.  |\obeylines| et |\obey!-spaces|
sont souvent utilis\'ees ensemble.
\example 
\obeylines
``Beware the Jabberwock, my son!!
\quad The jaws that bite, the claws that catch!!
Beware the Jubjub bird, and shun
\quad The frumious Bandersnatch!!''
|
\produces
\obeylines
``Beware the Jabberwock, my son!
\quad The jaws that bite, the claws that catch!
Beware the Jubjub bird, and shun
\quad The frumious Bandersnatch!''
\endexample
\enddesc

%\secondprinting{\vglue-\baselineskip\vskip0pt}

\begindesc
\easy\cts slash {}
\explain
Cette commande produit un ^{slash} (/) et pr\'evient aussi \TeX\ qu'il peut
couper la ligne apr\`es le slash, si n\'ecessaire.
\example
Her oldest cat, while apparently friendly to most people,
had a Jekyll\slash Hyde personality when it came to mice.
|
\produces
Her oldest cat, while apparently friendly to most people,
had a Jekyll\slash Hyde personality when it came to mice.
\endexample
\eix^^{coupures de ligne//encouragement ou d\'ecouragement}
\enddesc

%\secondprinting{\vfill\eject}


%==========================================================================
\subsection {param\`etres de coupure de lignes}

\begindesc
\bix^^{coupures de ligne//param\`etres affectant les}
%
\cts pretolerance {\param{nombre}}
\cts tolerance {\param{nombre}}
\explain
Ces param\`etres d\'eterminent la \minref{m\'ediocrit\'e} que \TeX\ tol\`erera
sur chaque ligne quand il choisit des coupures de ligne
pour un paragraphe.
La m\'ediocrit\'e est une mesure de combien l'espacement inter-mot d\'evie de
l'id\'eal.
|\pretolerance| sp\'ecifie la m\'ediocrit\'e tol\'erable pour
des coupures de ligne sans c\'esure~;
|\tolerance| sp\'ecifie la m\'ediocrit\'e tol\'erable pour des coupures de ligne 
avec c\'esure.
La m\'ediocrit\'e tol\'erable peut \^etre d\'epass\'ee de deux fa\c cons~:
un ligne est trop serr\'ee (les espaces inter-mots sont trop
petits) ou trop l\^ache (les espaces inter-mots sont trop grands).

\ulist
\li Si \TeX\ doit faire une ligne trop rel\^ach\'ee, il
se plaint d'un ``underfull hbox''.
\li Si \TeX\ doit faire une ligne trop resserr\'ee, 
il laisse la ligne d\'epasser dans la marge droite et
se plaint d'un ``overfull \minref{hbox}''.
\endulist

\noindent \TeX\ choisit des coupures de ligne selon les \'etapes suivantes~:
\olist
\li Il essaye de choisir des coupures de ligne sans c\'esures.
Si aucune des lignes r\'esultantes n'a de m\'edio\-crit\'e d\'epassant |\pretolerance|, 
les coupures de ligne sont accept\'ees et le paragraphe peut \^etre fait.
\li Sinon, il essaye un autre jeu de coupures de ligne, cette fois
en autorisant les c\'esures.  Si aucune des lignes r\'esultantes n'a une m\'edio\-crit\'e
d\'epassant |\tolerance|, le nouveau jeu de coupures de ligne est
acceptable et le paragraphe peut maintenant \^etre fait.
\li Autrement, il ajoute ^|\emergencystretch| (voir plus bas) \`a l'\'etire\-ment
de chaque ligne et essaye encore.
\li Si aucun de ces essais n'a produit de jeu de coupures de ligne acceptable,
il fait le paragraphe avec un ou plusieurs ``overfull hbox''
et s'en plaint.
\endolist

\PlainTeX\ initialise |\tolerance| \`a $200$ et |\pretolerance| \`a $100$.
Si vous mettez |\tolerance| \`a $10000$, \TeX\
devient infiniment tol\'erant et accepte tout espacement, quelque soit sa laideur
(\`a moins qu'il rencontre un mot qui ne tienne pas sur une ligne, m\^eme avec
c\'esure).  Ainsi en changeant |\tolerance| vous pouvez \'eviter des 
``overfull hbox'' et ``underfull hbox'', mais au prix de mauvais espacements.
En rendant |\pretolerance| plus grand vous pouvez faire que \TeX\ \'evite les 
c\'esure et s'ex\'ecute aussi plus rapidement,
mais, encore, au prix d'\'eventuels mauvais espacements.
Si vous mettez |\pretolerance| \`a $-1$,
\TeX\ n'essayera m\^eme pas de faire le paragraphe sans c\'esure.

Le param\`etre ^|\hbadness| (\xref \hbadness) d\'etermine le niveau de m\'ediocrit\'e
que \TeX\ tol\`erera avant  de se plaindre, mais |\hbadness| n'affecte pas
la mani\`ere dont \TeX\ compose votre document.
Le param\`etre ^|\hfuzz| (\xref \hfuzz) d\'etermine le montant dont
une hbox peut d\'epasser sa largeur sp\'ecifi\'ee avant que \TeX\ la consid\`ere 
erron\'ee.
\enddesc

\begindesc
\cts emergencystretch {\param{dimension}}
\explain
En mettant ce param\`etre sup\'erieur \`a z\'ero,
vous pouvez rendre plus facile \`a \TeX\ la
composition de votre document sans g\'en\'erer d'``overfull hbox''.
^^{overfull boxes}
C'est une meilleur alternative \`a |\tolerance=10000|,
car cela tend \`a produire des lignes r\'eellement laides.
Si \TeX\ ne peut pas composer un paragraphe sans d\'epasser ^|\tolerance|,
il tentera encore, en ajoutant |\emergencystretch| \`a l'\'etirement de chaque
ligne.
L'effet du changement est de r\'eduire la m\'ediocrit\'e de chaque ligne,
autorisant \TeX\ \`a faire des espaces plus larges qu'ils auraient \'et\'e autrement
et ainsi de choisir des coupures de ligne qui seront aussi
bons que possible selon les circonstances.
\enddesc

\begindesc
\cts looseness {\param{nombre}}
\explain
\minrefs{line break}
Ce param\`etre vous donne un moyen
de changer le nombre total de lignes dans un paragraphe par rapport \`a celui 
qu'il aurait \'et\'e de mani\`ere optimale.
|\looseness| est ainsi nomm\'e parce que c'est une mesure de combien perd le
paragraphe, c'est-\`a-dire, combien d'espace suppl\'ementaire il contient. 

Normalement, |\looseness| est \`a $0$ et
\TeX\ choisit des coupures de ligne selon sa mani\`ere habituelle.  Mais si
|\looseness| est \`a, disons, $3$, \TeX\ fait ainsi~:
\olist
\li Il choisit des coupures de ligne normalement, obtenant un paragraphe de $n$ lignes.
\li Il \'ecarte ces coupures de ligne et
tente de trouver un nouveau jeu de coupure de ligne 
qui donne le paragraphe en $n+3$ lignes.
(Sans l'\'etape pr\'ec\'edente, \TeX\ ne saurait pas la valeur de $n$.)
\li Si l'essai pr\'ec\'edent donne des lignes dont la m\'ediocrit\'e d\'epasse
|\tol!-er!-ance|,
^^|\tolerance|
il tente d'obtenir $n+2$ lignes---et si cela \'echoue aussi,
$n+1$ lignes, et finalement $n$ lignes encore.
\endolist
\noindent
De m\^eme, si |\looseness| est \`a $-n$,
\TeX\ tente de faire le paragraphe avec $n$ lignes de moins que la normale.
Le moyen le plus simple pour \TeX\ de faire un paragraphe plus long d'une ligne est de mettre
un seul mot sur la ligne en plus.  vous pouvez emp\^echer cela en
mettant un tilde (\xref{@not}) entre les deux derniers mots du paragraphe. 

Mettre |\looseness| est le meilleur moyen de forcer un paragraphe
\`a occuper un nombre de lignes donn\'e.
Le mettre \`a une valeur n\'egative est pratique quand vous essayez
d'augmenter la quantit\'e de texte que vous pouvez placer sur une page.
Similairement, le mettre \`a une valeur positive
est pratique quand vous essayez de diminuer la quantit\'e de texte sur 
une page.

\TeX\ remet |\looseness| \`a $0$ quand il termine un paragraphe, apr\`es avoir
coup\'e le paragraphe en lignes.
Si vous voulez changer le rel\^achement de plusieurs paragraphes, vous devez 
le faire individuellement pour chacun ou mettre le changement dans |\everypar|
\ctsref\everypar.
^^|\everypar//pour fixer \b\tt\\looseness\e|
\enddesc

\begindesc
\cts linepenalty {\param{nombre}}
\explain
\minrefs{line break}
Ce param\`etre sp\'ecifie les \minref{d\'em\'erites} que \TeX\ r\'epartit pour chaque coupure de ligne
quand il coupe un paragraphe en lignes.
La p\'enalit\'e est ind\'epen\-dante d'o\`u la coupure de ligne a lieu.
Augmenter la valeur
de ce param\`etre demande \`a \TeX\ d'essayer plus fortement de mettre un paragraphe avec 
un nombre minimum de lignes, m\^eme au co\^ut d'autres consid\'erations esth\'etiques
comme \'eviter des espacements inter-mots excessivement res\-serr\'es.
Les d\'em\'erites sont en unit\'es de \minref{m\'ediocrit\'e} au carr\'e, donc
vous devez assigner une valeur plut\^ot large \`a ce param\`etre (dans les
milliers) pour qu'elle ait un quelconque effet.
\PlainTeX\ met |\linepenalty| \`a $10$.
\enddesc

\begindesc
\cts adjdemerits {\param{nombre}}
\explain
\minrefs{line break}
^^{c\'esure//p\'enalit\'es pour}
{\tighten
Ce param\`etre sp\'ecifie des \minref{d\'em\'erites} additionnels que \TeX\ attache au
point de coupure entre deux lignes adjacentes qui sont
``incompatibles visuellement''. 
De telles paires de lignes font qu'un paragraphe appara\^\i t in\'egal.
Les incompatibilit\'es sont \'evalu\'ees en termes d'\'etroitesse ou de rel\^achement
de lignes~:
}
\olist\compact
\li Une ligne est resserr\'ee si son \minref{ressort} doit se r\'etr\'ecir d'au moins $50\%$.
\li Une ligne est d\'ecente si sa m\'ediocrit\'e est de $12$ ou moins.
\li Une ligne est rel\^ach\'ee si son \minref{ressort} doit s'\'etirer d'au moins $50\%$.
\li Une ligne est tr\`es rel\^ach\'ee si son ressort doit tellement s'\'etirer
que sa m\'ediocrit\'e d\'epasse $100$.
\endolist
Deux lignes adjacentes sont visuellement incompatibles
si leurs cat\'ego\-ries ne sont pas adjacentes, c'est-\`a-dire, une ligne resserr\'ee
est apr\`es une rel\^ach\'ee ou une ligne d\'ecente apr\`es une tr\`es rel\^ach\'ee.

Les d\'em\'erites sont en unit\'es de \minref{m\'ediocrit\'e} au carr\'ee, donc 
vous devez assigner une valeur plut\^ot large \`a ce param\`etre (dans les
milliers) pour qu'elle ait un quelconque effet.
\PlainTeX\ met |\adjdemerits| \`a~$10000$.
\enddesc

\begindesc
\bix^^{c\'esure//p\'enalit\'es pour}
\cts exhyphenpenalty {\param{nombre}}
\explain
\minrefs{line break}
Ce param\`etre sp\'ecifie la \minref{p\'enalit\'e} que \TeX\ attache \`a un 
point de coupure sur une c\'esure explicite comme celle de
``helter-skelter''.  Augmenter ce param\`etre a l'effet de d\'ecourager
\TeX\ de finir une ligne sur une c\'esure explicite.
\PlainTeX\ met |\exhyphenpenalty| \`a $50$.
\enddesc

\begindesc
\cts hyphenpenalty {\param{nombre}}
\explain
\minrefs{line break}
Ce param\`etre sp\'ecifie la \minref{p\'enalit\'e} que \TeX\ attache \`a un 
point de coupure sur une c\'esure implicite.
Des c\'esures implicites peuvent survenir du dictionnaire de c\'esure de \TeX\ ou
de ^{c\'esures optionnelles} que vous avez ins\'er\'e avec |\-|~(\xref{\@minus}).
^^|-//m\`ene \`a {\tt\\hyphenpenalty}|
Augmenter ce param\`etre a l'effet de d\'ecourager
\TeX\ de couper des mots.
\PlainTeX\ met |\hyphenpenalty| \`a $50$.
\enddesc

\begindesc
\cts doublehyphendemerits {\param{nombre}}
\explain
\minrefs{line break}
{\tighten
Ce param\`etre sp\'ecifie les \minref{d\'em\'erites} additionnels que \TeX\
attache \`a un point de coupure quand ce point de coupure arrive \`a
deux lignes cons\'ecu\-tives qui se terminent sur une c\'esure.
Augmenter la valeur de ce param\`etre d\'ecourage
\TeX\ de c\'esurer deux lignes \`a suivre.
Les d\'em\'erites sont en unit\'es de \minref{m\'ediocrit\'e} au carr\'e, donc
vous devez assigner une valeur plut\^ot large \`a ce param\`etre (dans les
milliers) pour qu'elle ait un quelconque effet.
\PlainTeX\ met |\doublehyphendemerits| \`a $10000$.
}
\enddesc

\begindesc
\cts finalhyphendemerits {\param{nombre}}
\explain
\minrefs{line break}
{\tighten
Ce param\`etre sp\'ecifie les \minref{d\'em\'erites} additionnels que \TeX\
attache \`a un point de coupure qui fait que l'avant derni\`ere ligne d'un
paragraphe se termine avec une c\'esure.
Une telle c\'esure est g\'en\'eralement consid\'er\'ee inesth\'etique
parce que l'espace blanc \'eventuel provoqu\'e par une derni\`ere ligne courte
la rend m\'eprisable.
Augmenter la valeur de ce param\`etre d\'ecourage
\TeX\ de finir l'avant derni\`ere ligne avec une c\'esure.
Les d\'em\'erites sont en unit\'es de \minref{m\'ediocrit\'e} au carr\'e, donc
vous devez assigner une valeur plut\^ot large \`a ce param\`etre (dans les
milliers) pour qu'elle ait un quelconque effet.
\PlainTeX\ met |\finalhyphendemerits| \`a $5000$.
}
\eix^^{c\'esure//p\'enalit\'es pour}
\enddesc

\begindesc
\cts binoppenalty {\param{nombre}}
\explain
^^{op\'erateurs}
Ce param\`etre sp\'ecifie la p\'enalit\'e d'une coupure d'une formule math\'e\-ma\-tique
apr\`es un op\'erateur binaire quand la formule appara\^\i t dans un paragraphe.
\PlainTeX\ met |\binoppenalty| \`a $700$.
\enddesc

\begindesc
\cts relpenalty {\param{nombre}}
\explain
^^{relations}
Ce param\`etre sp\'ecifie la p\'enalit\'e pour une coupure d'une formule math\'e\-ma\-tique
apr\`es une relation quand la formule appara\^\i t dans un paragraphe.
\PlainTeX\ met |\rel!-penal!-ty| \`a~$500$.

\eix^^{coupures de ligne//param\`etres affectant les}
\enddesc

%==========================================================================
\subsection {C\'esure}

\begindesc
\bix^^{c\'esure}
%
\easy\ctspecial - \ctsxrdef{@minus}
\explain
La commande |\-| ins\`ere une ``c\'esure optionnelle''
^^{c\'esures optionnelles}
dans un mot.
La c\'esure optionnelle autorise \TeX\ de couper le mot \`a cet
endroit.  \TeX\ n'est pas oblig\'e de couper l\`a---il ne le fera
que s'il le faut.  Cette commande est utile quand un mot 
qui appara\^\i t une ou deux fois dans votre document
doit \^etre coup\'e,
mais que \TeX\ ne peut pas trouver de point de coupure appropri\'ee de lui-m\^eme.
\example
Alice was exceedingly reluctant to shake hands first
with either Twee\-dle\-dum or Twee\-dle\-dee, for
fear of hurting the other one's feelings.
|
\produces
Alice was exceedingly reluctant to shake hands first
with either Twee\-dle\-dum or Twee\-dle\-dee, for
fear of hurting the other one's feelings.
\endexample
\enddesc

\begindesc
\cts discretionary {\rqbraces{\<pre-break text>}
   \rqbraces{\<post-break text>}
   \rqbraces{\<no-break text>}}
\explain
\minrefs{line break}
^^{c\'esure}
Cette commande sp\'ecifie une ``coupure optionnelle'', autrement dit,
un endroit o\`u \TeX\ peut couper une ligne.
Elle dit aussi \`a \TeX\ quel texte mettre de chaque cot\'e de la coupure.
\ulist
\li Si \TeX\ ne doit pas couper l\`a, il utilise le \<no-break text>.
\li Si \TeX\ doit couper l\`a, il met le \<pre-break text> juste avant
la coupure et le \<post-break text> juste apr\`es la coupure.
\endulist
\noindent
Comme avec |\-|,
\TeX\ n'est pas oblig\'e de couper une ligne sur une coupure optionnelle.
En fait, |\-| est normalement \'equivalent \`a |\discretionary!allowbreak{-}{}{}|.

Parfois, \TeX\ ins\`ere des coupures optionnelles de lui-m\^eme.
Par ex\-emple, il ins\`ere |\discretionary!allowbreak{}{}{}| apr\`es
une c\'esure explicite ou un tiret.

{\hyphenchar\tentt=-1 % needed to avoid weirdnesses
\example
% An ordinary discretionary hyphen (equivalent to \-):
\discretionary{-}{}{}
% A place where TeX can break a line, but should not
% insert a space if the line isn't broken there, e.g.,
% after a dash:
\discretionary{}{}{}
% Accounts for German usage: `flicken', but `flik-
% ken':
German ``fli\discretionary{k-}{k}{ck}en''
|
^^{c\'esure//allemande}
\endexample}

\enddesc

\begindesc
\cts hyphenation {\rqbraces{\<mot>\thinspace\vs\ $\ldots$\ \vs
   \thinspace\<mot>}}
\explain
\TeX\ garde un dictionnaire des exceptions de ses r\`egles de ^{c\'esure}.
Chaque entr\'ee du dictionnaire indique comment un mot particulier doit
\^etre coup\'e.  
La commande |\hyphenation| ajoute des mots au dictionnaire.
Son argument est une suite de mots s\'epar\'ee par des blancs. 
Les lettres majuscules et minuscules sont \'equivalentes.
Les c\'esures dans chaque mot indiquent les endroits
o\`u \TeX\ peut couper ce mot.
Un mot sans c\'esure ne sera jamais coup\'e.
Dans tous les cas, vous pouvez outrepasser le dictionnaire de c\'esure en
utilisant |\-| pour une occurrence particuli\`ere d'un mot.
Vous devez produire toutes les formes grammaticales d'un mot
que vous voulez que \TeX\ coupe, c'est-\`a-dire, le singulier et le pluriel.

\example
\hyphenation{Gry-phon my-co-phagy}
\hyphenation{man-u-script man-u-scripts piz-za}
|
\endexample
\enddesc

\begindesc
\cts uchyph {\param{nombre}}
\explain
Une valeur positive de |\uchyph| (c\'esure majuscule)
permet d\'ecouper des mots, comme des noms propres,
qui commencent par une lettre capitale.
Une valeur \`a z\'ero ou n\'egative
inhibe de telle coupure.  \PlainTeX\ met |\uchyph| \`a $1$,
donc \TeX\  essaye normalement de couper les mots qui commencent par une lettre capital.
\enddesc

\begindesc
\cts showhyphens {\rqbraces{\<mot>\thinspace\vs\ $\ldots$\ \vs
   \thinspace\<mot>}}
\explain
Cette commande n'est habituellement pas utilis\'ee dans des documents, 
mais vous pouvez l'utiliser sur votre terminal pour voir comment \TeX\ 
couperait certains jeux de mots al\'eatoires.
Les mots, avec les coupures visibles, apparaissent dans le log et sur
votre terminal.  Vous obtiendrez une plainte \`a propos d'une ``underfull 
hbox''---ignorez la.
\example
\showhyphens{threshold quizzical draughts argumentative}
|
\logproduces
Underfull \hbox (badness 10000) detected at line 0
[] \tenrm thresh-old quizzi-cal draughts ar-gu-men-ta-tive
|
\endexample
\enddesc

\begindesc
\cts language {\param{number}}
\explain
Des langages diff\'erents ont diff\'erents jeux de r\`egles de c\'esure.
Ce para\-m\`etre d\'etermine le jeu de ^{r\`egles de c\'esure} que \TeX\ utilise.
En changeant |\language| vous pouvez obtenir que \TeX\
coupe des portions de texte ou des documents entiers en accord avec les 
r\`egles de c\'esure appropri\'ees \`a un langage particulier.
^^{langues europ\'eennes}
Votre ^{support local} sur \TeX\ vous dira si des
jeux additionnels de r\`egles de c\'esure sont accessibles (derri\`ere celui
pour l'anglais)
et quelles sont les valeurs appropri\'ees de |\language|.
La valeur par d\'efaut de |\language| est $0$.

\TeX\ met le langage courant \`a $0$ au d\'ebut de tous les paragraphes,
et compare |\language| au langage courant \`a chaque fois qu'il ajoute
un caract\`ere au paragraphe courant.
Si ce n'est pas le m\^eme, \TeX\ ajoute un ^{\elementextra} indiquant 
que le langage change.
Cet \'el\'ement extraordinaire est la preuve pour les prochaines ex\'ecutions que les r\`egles de langage
peuvent changer.
\enddesc

\begindesc
\cts setlanguage {\<nombre>}
\explain
Cette commande met le langage courant \`a \<nombre>
en ins\'erant le m\^eme \'el\'ement extraordinaire que vous obtiendrez en changeant 
^|\language|.
En outre, elle ne change pas la valeur de |\language|.
\enddesc

\begindesc
\cts lefthyphenmin {\param{nombre}}
\cts righthyphenmin {\param{nombre}}
\explain
Ces param\`etres sp\'ecifient les plus petits fragments de mots que \TeX\ autorise
\`a gauche et \`a droite d'un mot coup\'e.
\PlainTeX\ les initialise par d\'efaut \`a $2$ et $3$ respectivement~;
ce sont les valeurs recommand\'ees pour l'anglais.
\enddesc

\begindesc
\bix^^{polices//caract\`eres de c\'esure pour}
\cts hyphenchar {\<police>\param{nombre}}
\explain
\TeX\ n'utilise pas n\'ecessairement le caract\`ere `-' aux points de c\'esure.
A la place, il utilise le |\hyphenchar| de la police courante, qui est 
habituellement `-' mais pas n\'ecessairement. Si une police a une valeur
 |\hyphenchar| n\'egative,
\TeX\ ne coupe pas de mots dans cette police.

Notez que \<police> est une s\'equence de contr\^ole
qui nomme une police, pas un \<nom de police> qui nomme des fichiers de
police.
Attention~: 
un assignement \`a |\hyphenchar| n'est \emph{pas} r\'einitialis\'e \`a la fin
d'un groupe.
Si vous voulez changer |\hyphenchar| localement, vous devrez 
sauvegarder et restaurer sa valeur originale explicitement.

\example
\hyphenchar\tenrm = `- 
   % Set hyphenation for tenrm font to `-'.
\hyphenchar\tentt = -1
   % Don't hyphenate words in font tentt.
|
\endexample
\enddesc

\begindesc
\cts defaulthyphenchar {\param{nombre}}
\explain
Quand \TeX\ lit le fichier de m\'etriques
^^{fichiers de m\'etriques//tiret de c\'esure par d\'efaut dans}
pour une police en r\'eponse \`a une commande
^|\font|, il met le ^|\hyphenchar| de la police dans
|\default!-hyphen!-char|.
Si la valeur de |\default!-hyphen!-char| n'est
pas dans la fourchette $0$--$255$ quand vous chargez une police,
\TeX\ ne coupera aucun mot dans cette police \`a moins que vous
outrepassiez la d\'ecision en mettant le |\hyphenchar| de la police apr\`es.
\PlainTeX\  met |\default!-hyphen!-char| \`a $45$, le code \ascii\ 
pour `|-|'.
\example
\defaulthyphenchar = `-
   % Assume `-' is the hyphen, unless overridden.
\defaulthyphenchar = -1
   % Don't hyphenate, unless overridden.
|
\endexample

\eix^^{polices//caract\`eres de c\'esure pour}
\enddesc

\see |\pretolerance| (\xref \pretolerance).
\eix^^{c\'esure}
\eix^^{coupures de ligne}

%==========================================================================
\section {Ent\^etes de section, listes et th\'eor\`emes}

\begindesc
^^{ent\^etes de section}
\easy\ctspecial beginsection {\<argument>\thinspace{\bt\\par}}
   \ctsxrdef{@beginsection}
\explain
Vous pouvez utiliser cette commande pour d\'ebuter une subdivision majeure de votre document.
\<argument> est pr\'evu pour servir de titre de section.
|\beginsection| amplifie \<argument>
par de l'espace vertical suppl\'ementaire et le met en
police grasse, justifi\'e \`a gauche.
Vous pouvez produire le |\par| qui termine \<argument> avec une ligne blanche.
\let\message = \gobble % Don't bother to tell us about Pig and Pepper.
\example
$\ldots$  till she had brought herself down to nine
inches high.

\beginsection Section 6. Pig and Pepper

For a minute or two she stood looking at the house $\ldots$
|
\produces
$\ldots$  till she had brought herself down to nine
inches high.

\beginsection Section 6. Pig and Pepper

For a minute or two she stood looking at the house $\ldots$
\endexample
\enddesc

\begindesc
\cts item {\<argument>}
\cts itemitem {\<argument>}
\explain
^^{listes d'\'el\'ements}
Cette commande est utile pour cr\'eer des ^{listes d'\'el\'ements}.  Tout le paragraphe
suivant \<argument> est indent\'e de |\parindent|
^^|\parindent//indentation pour listes d'\'el\'ements|
(pour |\item|) ou par |2\parindent| (pour |\itemitem|).
(Voir \xrefpg{\parindent} pour une explication de |\parindent|.)
Ensuite \<argument>,
suivi par un espace demi-cadratin, est plac\'e juste \`a
la gauche du texte de la
premi\`ere ligne du paragraphe pour qu'il tombe avec l'indentation du paragraphe
comme sp\'ecifi\'e par |\parindent|.

Si vous voulez inclure plus d'un
paragraphe dans un \'el\'ement, mettez |\item{}| devant les paragraphes additionnels.
\example
{\parindent = 18pt
\noindent Here is what we require:
\item{1.}Three eggs in their shells,
but with the yolks removed.
\item{2.}Two separate glass cups containing:
\itemitem{(a)}One-half cup {\it used} motor oil.
\itemitem{(b)}One cup port wine, preferably French.
\item{3.}Juice and skin of one turnip.}
|
\produces
{\parindent = 18pt
\noindent Here is what we require:
\item{1.}Three eggs in their shells,
but with the yolks removed.
\item{2.}Two separate glass cups containing:
\itemitem{(a)}One-half cup {\it used} motor oil.
\itemitem{(b)}One cup port wine, preferably French.
\item{3.}Juice and skin of one turnip.}
\endexample
\enddesc

\begindesc
\easy\ctspecial proclaim {\<argument>{\tt.}\vs\thinspace
   \<texte g\'en\'eral>\thinspace{\bt\\par}}
   \ctsxrdef{@proclaim}
\explain
^^{th\'eor\`emes}
^^{lemmes}
^^{hypoth\`eses}
Cette commande ``proclame'' un th\'eor\`eme, un lemme, une hypoth\`ese, etc.
Elle met \<argument> en police grasse et le paragraphe suivant en
italique.  \<arg\-u\-ment> doit \^etre suivi par un point et un token espace,
qui servent \`a s\'eparer \<argument> du \<texte g\'en\'eral>.
\<texte g\'en\'eral> est constitu\'e du texte jusqu'\`a la fronti\`ere du paragraphe suivant,
sauf que vous pouvez inclure des paragraphes multiples en les mettant
entre accolades et terminer un paragraphe apr\`es l'accolade droite fermante.
\example
\proclaim Theorem 1.
What I say is not to be believed.

\proclaim Corollary 1. Theorem 1 is false.\par
|
\produces
\proclaim Theorem 1.
What I say is not to be believed.

\proclaim Corollary 1. Theorem 1 is false.\par
\endexample
\enddesc

\enddescriptions
\endchapter
\byebye

