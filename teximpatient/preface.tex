% -*- coding: utf-8-unix; mode: tex -*-
% This is part of the book TeX for the Impatient.
% Copyright (C) 2003 Paul W. Abrahams, Kathryn A. Hargreaves, Karl Berry.
% See file fdl.tex for copying conditions.

\input macros
\frontchapter{Preface}
%%% ORIGINAL BEGIN %%%
% {\tighten
% Donald Knuth's \TeX, a computerized typesetting system,
% provides nearly everything
% needed for high-quality typesetting of mathematical
% notations as well as of ordinary text.
% It is particularly notable for its flexibility, its superb hyphenation, and its
% ability to choose aesthetically satisfying line breaks.
% Because
% of its extraordinary capabilities, \TeX\ has become the leading typesetting
% system for mathematics, science, and engineering and has been adopted as
% a standard by the American Mathematical Society.  A companion program,
% ^{\Metafont},
% can construct arbitrary letterforms including, in particular, any symbols that
% might be needed in mathematics.
% Both \TeX\ and \Metafont\ are widely available within the
% scientific and engineering community and have been implemented on a
% variety of computers.
% \TeX\ isn't perfect---it lacks integrated support for graphics, and
% some effects such as ^{revision bars} are very difficult to produce---%
% but these drawbacks are far outweighed by its advantages.
% \par}
%%% ORIGINAL END %%%
%%% PTBR BEGIN %%%
{\tighten \TeX{} , feito por Donald Knuth, um sistema tipográfico
computadorizado, fornece praticamente tudo necessário para uma tipografia de
alta qualidade de notações matemáticas bem como de texto ordinário. É
particularmente notável por sua flexibilidade, sua hifenação suprema, e sua
capacidade de selecionar quebras de linha esteticamente satisfatórias. Por causa
de suas extraordinárias capacidades, \TeX{} tornou-se o sistema tipográfico líder
para matemática, ciência e engenharia, e tem sido adotado como padrão pela {\it
American Mathematical Society}. Um programa acompanhante, ^{\Metafont}, pode
construir fontes arbitrárias, incluindo, em particular, quaisquer símbolos que
podem ser necessários em matemática. Ambos \TeX{} e \Metafont\ estão largamente
disponíveis na comunidade científica e de engenharia e têm sido implementados em
uma variedade de computadores. \TeX{} não é perfeito -- falta a ele suporte
integrado a gráficos, e alguns efeitos tais como ^{barras de revisão} são bem
difíceis de produzir --- mas estas desvantagens são superadas de longe pelas
suas vantagens. \par}
%%% PTBR END %%%
%%% ORIGINAL BEGIN %%%
% \thisbook\/ is intended to serve scientists, mathematicians, and
% technical typists for whom \TeX\ is a useful tool rather than a primary
% interest, as well as computer people who have a strong interest in \TeX\
% for its own sake.  We also intend it to serve both newcomers to \TeX\
% and those who are already familiar with \TeX.  We assume that our
% readers are comfortable working with computers and that they want to get
% the information they need as quickly as possible.  Our aim is to provide
% that information clearly, concisely, and accessibly.
%%% ORIGINAL END %%%

%%% PTBR BEGIN %%%
\thisbook\/ é projetado para servir a cientistas, matemáticos e tipistas
técnicos para os quais \TeX{} é uma ferramenta útil em vez de um interesse
primário, bem como usuários de computadores que tenham forte interesse em \TeX{}
por razões próprias. Nós também pretendemos que ele sirva a ambos, iniciantes em
\TeX{} e aqueles já familiares com o \TeX{} . Nós assumimos que nossos leitores
estão confortáveis trabalhando com computadores e querem obter a informação que
precisam o mais rápido possível. Nossa meta é fornecer esta informação de
maneira clara, concisa e acessível.
%%% PTBR END %%%

%%% ORIGINAL BEGIN %%%
% {\tighten This book therefore provides a bright searchlight, a stout
% walking-stick, and detailed maps for exploring and using \TeX.  It will
% enable you to master \TeX\ at a rapid pace through inquiry and
% experiment, but it will not lead you by the hand through the entire
% \TeX\ system.  Our approach is to provide you with a handbook for \TeX\
% that makes it easy for you to retrieve whatever information you need.
% We explain both the full repertoire of \TeX\ commands and the concepts
% that underlie them.  You won't have to waste your time plowing through
% material that you neither need nor want.  \par}
%%% ORIGINAL END %%%
%%% PTBR BEGIN %%%
{\tighten Este livro, portanto, fornece um holofote, uma robusta bengala, e
mapas detalhados para explorar e usar \TeX{}. Ele te habilitará a dominar \TeX{}
em um ritmo rápido mediante investigação e experimentação, mas não vai te levar
pela mão por todo o sistema \TeX{}. Nossa abordagem é te fornecer um manual para
\TeX{} que torna fácil a você recuperar qualquer informação que precisar. Nós
explicamos o repertório de comandos \TeX{} e conceitos que os embasam. Não temos
que gastar seu tempo cavoucando por material que você não precisa e nem
quer. \par}
%%% PTBR END %%%

%%% ORIGINAL BEGIN %%%
% In the early sections we also provide you with enough orientation so
% that you can get started if you haven't used \TeX\ before.  We assume
% that you have access to a \TeX\ implementation and that you know how to
% use a text editor, but we don't assume much else about your background.
% Because this book is organized for ready reference, you'll continue to
% find it useful as you become more familiar with \TeX.  If you prefer to
% start with a carefully guided tour, we recommend that you first read
% Knuth's ^{\texbook} (see \xrefpg{resources} for a citation), passing
% over the ``dangerous bend'' sections, and then return to this book for
% additional information and for reference as you start to use \TeX.  (The
% dangerous bend sections of \texbook\ cover advanced topics.)
%%% ORIGINAL END %%%
%%% PTBR BEGIN %%%
Nas primeiras seções nós também te fornecemos informação suficiente para que
possa começar caso não tenha usado \TeX{} antes. Nós assumimos que você tenha
acesso a uma implemantação \TeX{} e que você saiba como usar um editor de
textos, mas não assumimos muito mais acerca do seu conhecimento prévio. Como
este livro é organizado para pronta referência, ele continuará a ser útil à
medida em que você tornar-se mais familiar com o \TeX{}. Se você prefere começar
com uma jornada mais cuidadosamente guiada, te recomendamos que leia primeiro o
^{\texbook} de Knuth (veja \xrefpg{resources} para uma citação), ignorando as
seções mais ``perigosas'', e então retornando para este livro para informação
adicional e para referência enquanto começa a usar \TeX{}. (As seções perigosas
do \texbook\ cobrem tópicos avançados.)
%%% PTBR END %%%

%%% ORIGINAL BEGIN %%%
% The structure of \TeX\ is really quite simple: a \TeX\ input document
% consists of ordinary text interspersed with commands that give \TeX\
% further instructions on how to typeset your document.  Things like math
% formulas contain many such commands, while expository text contains
% relatively few of them.
%%% ORIGINAL END %%%
%%% PTBR BEGIN %%%
A estrutura de \TeX{} é realmente bastante simples: um documento de entrada
\TeX\ consiste de um texto ordinário intercalado com comandos que dão ao \TeX{}
instruções adicionais de como formatar seu documento. Coisas como fórmulas
matemáticas contêm muitos destes comandos, enquanto textos expositórios contêm
relativamente poucos.
%%% PTBR END %%%

%%% ORIGINAL BEGIN %%%
% The time-consuming part of learning \TeX\ is learning the commands and
% the concepts underlying their descriptions.  Thus we've devoted most of
% the book to defining and explaining the commands and the concepts.
% We've also provided examples showing \TeX\ typeset output and the
% corresponding input, hints on solving common problems, information about
% error messages, and so forth.  We've supplied extensive cross-references
% by page number and a complete index.
%%% ORIGINAL END %%%
%%% PTBR BEGIN %%%
A parte do aprendizado de \TeX{} que demanda mais tempo é aprender os comandos e
os conceitos subjacentes às suas descrições. Por isso, nós devotaremos a maior
parte do livro definindo e explanando os comandos e conceitos. Nós também
fornecemos exemplos mostrando a saída formatada e a entrada correspondente,
dicas sobre resolver problemas comuns, informações sobre mensagens de erro, e
assim por diante. Nós suprimos referências extensivas por número de página e um
índice completo.
%%% PTBR END %%%

%%% ORIGINAL BEGIN %%%
% We've arranged the descriptions of the commands so that you can look
% them up either by function or alphabetically.  The functional
% arrangement is what you need when you know what you want to do but you
% don't know what command might do it for you.  The alphabetical arrangement
% is what you need when you know the name of a command but you don't know exactly
% what it does.
%%% ORIGINAL END %%%
%%% PTBR BEGIN %%%
Nós arranjamos as descrições dos comandos de tal forma que você possa
procurá-los seja por funcionalidade seja alfabeticamente. O arranjo funcional é
o que você precisa quando sabe o que quer fazer mas não sabe que comando pode
servir. O arranjo alfabético é o que você precisa quando sabe o nome do comando
mas não sabe exatamente o que ele faz.
%%% PTBR END %%%

%%% ORIGINAL BEGIN %%%
% We must caution you that we haven't tried to provide a complete
% definition of \TeX.  For that you'll need ^{\texbook}, which is the
% original source of information on \TeX.  \texbook\ also contains a lot
% of information about the fine points of using \TeX, particularly on the
% subject of composing math formulas.  We recommend it highly.
%%% ORIGINAL END %%%
%%% PTBR BEGIN %%%
Devemos te alertar que não tentamos prover uma definição completa de
\TeX{}. Para isso você precisa do ^{\texbook} que é a fonte original de
informação sobre \TeX{}. \texbook\ também contém bastante informação sobre os
pontos finos de usar \TeX{}, particularmente na questão de compor fórmulas
matemáticas. Nós o recomendamos grandemente.
%%% PTBR END %%%

%%% ORIGINAL BEGIN %%%
% In 1989 Knuth made a major revision to \TeX\ in order to adapt it to
% $8$-bit character sets, needed to support typesetting for languages
% other than English.  The description of \TeX\ in this book incorporates
% that revision (see \xref{newtex}).
%%% ORIGINAL END %%%
%%% PTBR BEGIN %%%
Em 1989, Knuth vez uma grande revisão no \TeX{} a fim de adaptá-lo para
conjuntos de caracteres de $8$ bits, necessário para suportar a formatação para
linguagens diferentes do inglês. A descrição de \TeX{} neste livro incorpora
esta revisão (veja \xref{newtex}).
%%% PTBR END %%%

%%% ORIGINAL BEGIN %%%
% {\tighten You may be using a specialized form of \TeX\ such as ^{\LaTeX}
% or ^{\AMSTeX} (see \xref{resources}).  Although these specialized forms
% are self-contained, you may still want to use some of the facilities of
% \TeX\ itself now and then in order to gain the finer control that only
% \TeX\ can provide.  This book can help you to learn what you need to
% know about those facilities without having to learn about a lot of other
% things that you aren't interested~in.  \par}
%%% ORIGINAL END %%%
%%% PTBR BEGIN %%%
{\tighten Você pode usar uma forma especializada de \TeX{} tal como ^{\LaTeX} ou
  ^{\AMSTeX} (veja \xref{resources}). Apesar de estas formas especializadas
  serem auto-contidas, você pode ainda querer utilizar algumas das ferramentas
  do próprio \TeX{} de agora em diante a fim de obter o controle fino que apenas
  o \TeX{} pode proporcionar. Este livro pode te ajudar a aprender o que você
  precisa saber sobre estas ferramentas sem ter que aprender tantas outras
  coisas que não estiver interessado. \par}
%%% PTBR END %%%

%%% ORIGINAL BEGIN %%%
% Two of us (K.A.H. and K.B.) were generously supported by the
% University of Massachusetts at Boston during the preparation of this
% book.  In particular, Rick Martin kept the machines running, and
% Robert~A. Morris and Betty O'Neil made the machines available.  Paul
% English of Interleaf helped us produce proofs for a cover design.
%%% ORIGINAL END %%%
%%% PTBR BEGIN %%%
Dois de nós (K.A.H. and K.B.) foram generosamente suportados pela {\it
  University of Massachusetts} em Boston durante a preparação deste livro. Em
particular, Rick Martin manteve as máquinas funcionando, e Robert~A. Morris e
Betty O'Neil disponibilizou as máquinas.  Paul English do {\it Interleaf} nos
ajudou a produzir provas para um modelo de capa.
%%% PTBR END %%%

%%% ORIGINAL BEGIN %%%
% We wish to thank the reviewers of our book: Richard Furuta of the
% University of Maryland, John Gourlay of Arbortext, Inc., Jill Carter
% Knuth, and Richard Rubinstein of the Digital Equipment Corporation. We
% took to heart their perceptive and unsparing criticisms of the original
% manuscript, and the book has benefitted greatly from their insights.
%%% ORIGINAL END %%%
%%% PTBR BEGIN %%%
Nós desejamos agradecer aos revisores de nosso livro: Richard Furuta da {\it
  University of Maryland}, John Gourlay do {\it Arbortext, Inc.}, Jill Carter
Knuth, e Richard Rubinstein da {\it Digital Equipment Corporation}. Recebemos de
coração suas perceptivas e generosas críticas ao manuscrito original, e o livro
beneficiou-se grandiosamente de suas percepções.
%%% PTBR END %%%

%%% ORIGINAL BEGIN %%%
% We are particularly grateful to our editor, Peter Gordon of
% Addison-Wesley.  This book was really his idea, and throughout its
% development he has been a source of encouragement and valuable
% advice.  We thank his assistant at Addison-Wesley, Helen Goldstein, for
% her help in so many ways, and Loren Stevens of Addison-Wesley for her
% skill and energy in shepherding this book through the production
% process.  Were it not for our copyeditor, Janice Byer, a number of small
% but irritating errors would have remained in this book.  We appreciate
% her sensitivity and taste in correcting what needed to be corrected
% while leaving what did not need to be corrected alone.  Finally, we wish
% to thank Jim Byrnes of Prometheus Inc. for making this collaboration
% possible by introducing us to each other.
%%% ORIGINAL END %%%
%%% PTBR BEGIN %%%
Nós somos particularmente gratos por nosso editor, Peter Gordon da
Addison-Wesley. Este livro realmente foi ideia dele, e ao longo do
desenvolvimento ele foi fonte de encorajamento e valioso aconselhamento. Nós
agradecemos sua assistente na Addison-Wesley, Helen Goldstein, por sua ajuda de
tantas formas, e Loren Stevens da Addison-Wesley por sua perícia e energia em
guiar o processo de edição. Não fosse nossa editora de cópia, Janice Byer, um
número de pequenos mas irritantes erros teria permanecido no livro. Nós
apreciamos sua sensibilidade e gosto em corrigir o que precisava ser corrigido
enquanto deixava o que não precisava ser corrigido em seu lugar. Finalmente,
desejamos agradecer a Jim Byrnes da Prometheus Inc. por tornar esta colaboração
possível ao apresentar-nos um ao outro.
%%% PTBR END %%

\vskip1.5\baselineskip

\line{\it Deerfield, Massachusetts\hfil\rm P.\thinspace W.\thinspace A.}
\line{\it Manomet, Massachusetts\hfil\rm K.\thinspace A.\thinspace H.,
  K.\thinspace B.}

\vskip2\baselineskip
%%% ORIGINAL BEGIN %%%
% \noindent {\bf Preface to the free edition:} This book was originally
% published in 1990 by Addison-Wesley.  In 2003, it was declared out of
% print and Addison-Wesley generously reverted all rights to us, the
% authors.  We decided to make the book available in source form, under
% the GNU Free Documentation License, as our way of supporting the
% community which supported the book in the first place.  See the
% copyright page for more information on the licensing.
%%% ORIGINAL END %%%
%%% PTBR BEGIN %%%
\noindent {\bf Prefácio à Edição Livre:} Este livro foi originalmente publicado
em 1990 pela Addison-Wesley. Em 2003, foi declarado fora de impressão e a
Addison-Wesley generosamente reverteu todos os direitos para nós, os
autores. Nós decidimos tornar o livro disponível em forma de código-fonte, sob a
licença {\it GNU Free Documentation License}, como nossa forma de apoiar a
comunidade que apoiou o livro em primeiro lugar. Veja a página de direitos de
cópia para mais informação sobre o licenciamento.
%%% PTBR END %%%

%%% ORIGINAL BEGIN %%%
% The illustrations which were part of the original book are not included
% here.  Some of the fonts have also been changed; now, only
% freely-available fonts are used.  We left the cropmarks and galley
% information on the pages, to serve as identification.  An old version of
% Eplain was used to produce it; see the {\tt eplain.tex} file for
% details.
%%% ORIGINAL END %%%
%%% PTBR BEGIN %%%
As ilustrações que eram parte do livro original não estão inclusas aqui. Algumas
das fontes também foram modificadas; agora, somente fontes livremente
disponíveis são usadas. Nó deixamos as marcas de corte e informação de galé nas
páginas, para servir como identificação. Uma antiga versão do Eplain foi usada
para produzi-la; veja o arquivo {\tt eplain.tex} para detalhes.
%%% PTBR END %%%

%%% ORIGINAL BEGIN %%%
% We don't plan to make any further changes or additions to the book
% ourselves, except for correction of any outright errors reported to us,
% and perhaps inclusion of the illustrations.
%%% ORIGINAL END %%%
%%% PTBR BEGIN %%%
Nós mesmos não planejamos realizar mudanças ou adições extras ao livro, exceto
para correção de quaisquer erros flagrantes reportados para nós, e talvez
inclusão de ilustrações.
%%% PTBR END %%%

%%% ORIGINAL BEGIN %%%
% Our distribution of the book is at {\tt
% ftp://tug.org/tex/impatient}.  You can reach us by email at {\tt
% impatient@tug.org}.
%%% ORIGINAL END %%%
%%% PTBR BEGIN %%%
Nossa distribuição do livro está em {\tt ftp://tug.org/tex/impatient}. Você pode
nos encontrar por e-mail em {\tt impatient@tug.org}.
%%% PTBR END %%%

\pagebreak
\byebye
