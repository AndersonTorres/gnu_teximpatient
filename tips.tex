% This is part of the book TeX for the Impatient.
% Copyright (C) 2003 Paul W. Abrahams, Kathryn A. Hargreaves, Karl Berry.
% See file fdl.tex for copying conditions.

\input macros
\chapter{Tips and techniques}

\chapterdef{tips}

\TeX\ is a complex program that occasionally works its will in
mysterious ways.  In this section we offer some tips on solving
problems that you might encounter and explain some handy techniques.


\section  Correcting bad page breaks

\bix^^{page breaks//bad}
Sometimes \TeX\ breaks a page right in the middle of material that you
want to keep together---for example,
a section heading and the text that follows it,
or a short list of related items.  There are two ways to correct the
situation:
\ulist\compact
\li You can force the material to be kept together.
\li You can force a page break at a different place.
\endulist

The simplest way to force \TeX\ to keep material together on a page is
to enclose the material in a vbox using the |\vbox| command \ctsref{\vbox}.
^^|\vbox//fixing page breaks with|
A vbox is ordinarily better than an hbox for this purpose because most
often the material to be kept together,
e.g., a sequence of paragraphs, will be vertical mode material.
You should
precede and follow the vbox by an implicit or explicit paragraph command
(either a blank line or |\par|); otherwise 
\TeX\ may try to make the vbox part of an an adjacent paragraph.
The vbox method has an important limitation:
you can't apply it to portions of text smaller than a paragraph.

You can sometimes keep the lines of a single paragraph together by enclosing
the paragraph in a group and assigning |\interlinepenalty|
\ctsref{\interlinepenalty} 
a value of $10000$ at the start of the group (or elsewhere before the
end of the paragraph).
This method causes \TeX\ to consider page breaks within that paragraph to be
infinitely undesirable.
However, if all the page breaks that \TeX\ can find are infinitely undesirable,
it may break the page within the paragraph anyway.

A ^|\nobreak| command  (\xref{vnobreak})
after the end of a paragraph prevents \TeX\
from breaking the page at the following item
(unless that item happens to be a penalty of less than $10000$).
This is also the best way to prevent a page break after a heading,
since a heading usually behaves like a paragraph.
The |\nobreak| must follow the blank line or |\par| that ends
the paragraph so that \TeX\ won't treat the |\nobreak| as part of the
paragraph.
For the |\nobreak| to be effective, it must also come before
any legal breakpoint at the end of the paragraph.
The glue that \TeX\ inserts
before the next paragraph is such a breakpoint,
and so is any vertical glue that you insert explicitly after a paragraph.
Thus the |\nobreak| should usually be
the very first thing after the end of the paragraph or heading.

You can use the ^|\eject| command \ctsref{\eject}
to force \TeX\ to break a page at a
particular place.  Within a paragraph, you can use the combination 
`|\vadjust{\vfill\eject}|' 
\ctsref{\vadjust}
^^|\vadjust|
to force a break after the next complete output line.
The reason for preceding |\eject| by ^|\vfill| 
\ctsref{\vfill} is to
get \TeX\ to fill out the page with blank space.  
However, using |\eject| to fix page break problems
has a major disadvantage:
if the page boundaries in your document change,
the page breaks that you've inserted may no longer be where you want them.

If you don't provide \TeX\ with a |\vfill| command to fill out the page
after an |\eject|,
\TeX\ redistributes the extra blank space as best it can and then usually
complains that ``an
underfull |\vbox| (badness $10000$) has occurred while |\output| is active.''
You may encounter a similar problem with
any of the methods mentioned above for enclosing
material that you want to keep together.

The ^|\filbreak| command \ctsref\filbreak{}
\xrdef{filbreak}
provides a way of keeping the lines of one or more paragraphs
(or other vertical mode material) together on
a page.  If you enclose a paragraph in |\filbreak|s, \TeX\ will effectively
ignore the |\filbreak|s if the paragraph fits on the current page and
break the page before the first |\filbreak| if the paragraph doesn't fit.
If you put |\filbreak|s around each paragraph in a sequence of paragraphs,
like this:
{\obeylines\display{
{\tt \\filbreak}
\<paragraph>
{\tt \\filbreak}
\<paragraph>
{\tt \\filbreak}
\leavevmode\indent\vdots
\<paragraph>
{\tt \\filbreak}
}}
\vfil\eject
\noindent
\TeX\ will keep the lines of each paragraph together on a page.
If \TeX{} breaks a page at a |\filbreak|, it will fill the bottom of the
page with blank space.

Sometimes you can get \TeX\ to modify the length of a page by changing
the ^|\looseness| parameter \ctsref\looseness{} for one or more paragraphs.
Setting |\looseness| negative within a paragraph causes \TeX\ to try to squeeze
the paragraph into fewer lines; setting it positive causes \TeX\ to try to
expand the paragraph into more lines.  The disadvantage of changing |\looseness|
is that the interword spacing in the affected region won't be optimal.
You can get further information about \TeX's attempted line breaks by
setting |\tracingpages| (\xref\tracingpages) to $1$.


\section Preserving the end of a page

Sometimes you need to modify something on a single page and you want to avoid
reprinting the entire document.  If your modification doesn't 
change the page length too much, there's hope.  You need to fix the end
of the page so that it falls in the same place;
the methods are similar to the ones for fixing a bad page break.

If the original end of page came between paragraphs, you can
force a page break at the same place
using any of the methods we've described above.  Otherwise,
you must force \emph{both} a line break and a page break at a particular
place.  If the new page is shorter than the old one, the sequence:
\csdisplay
\vadjust{\vfill\eject}\break
|
^^|\vadjust|
should do the trick.
But if the new page is longer,
the problem is far more difficult because \TeX\ has probably already
squeezed the page as tightly as it can.  Your only hopes in this case are
to set ^|\looseness| \ctsref\looseness{} to a negative value,
to shorten some of the vertical skips on the page, to add some shrink to
^|\parskip| (\xref\parskip) if it was nonzero, or, as a last resort,
to decrease |\baselineskip| \ctsref{\baselineskip} ever so slightly.
^^|\baselineskip//and preserving end of page|

\section Leaving space at the top of a page

\null
^^{vertical space//reserving at top of page}
You can usually use the |\vskip| command \ctsref\vskip{}
to leave vertical space on a page.
That doesn't work at the top of a page, however, since
\TeX\ discards glue, kerns, and penalties that occur just after a page break.
^^{page breaks//glue at}
Use the ^|\topglue| command \ctsref\topglue\ instead; it produces glue that
never disappears.
\eix^^{page breaks//bad}



\section Correcting bad line breaks

\null
\bix^^{line breaks//bad}
If \TeX\ breaks a line in the middle of material that you wanted to keep
on a single line, there are several ways to correct the situation:
\ulist\compact
\li You can force a break in a nearby place with the
|\break| command (\xref{hbreak}).
^^|\break//correcting line breaks with|
\li You can insert a tie (|~|) between two words
(see \xref{@not}) to prevent a break between them.\ttidxref{~}
\li You can tell \TeX\ about hyphenations that it wouldn't 
otherwise consider by
inserting one or more discretionary hyphens in various words
(see |\-|, \xref{\@minus}).
\ctsidxref{-//in line breaking}
^^{discretionary hyphens//bad line breaks, correcting with}
\li You can enclose several words in an hbox using the ^|\hbox|
command \ctsref{\hbox}.
\endulist

The disadvantage of all of these methods, except for inserting discretionary
hyphens, is that they may make it impossible for \TeX\ to find a
satisfactory set of line breaks.  Should that happen, \TeX\ will
set one or more underfull or overfull boxes and complain about it.
^^{underfull boxes} ^^{overfull boxes}
The hbox method has a further disadvantage: because \TeX\ sets an hbox
as a single unit without considering its context, the interword space 
within the hbox may not be consistent with the interword space in the rest of
the line.
\eix^^{line breaks//bad}


\section Correcting overfull or underfull boxes

\null
\bix^^{overfull boxes}
\bix^^{underfull boxes}
\bix^^{boxes//overfull}
\bix^^{boxes//underfull}
If \TeX\ complains about an overfull box, it means you've put
more material into a box than that box has room for.  
Similarly, if \TeX\ complains about an underfull box, it means you
haven't put enough material into the box.  You can encounter
these complaints under many different circumstances, so let's look at the more
common ones:

\ulist
\li An overfull hbox that's a line of a paragraph indicates that the line was
too long and that 
\TeX\ couldn't rearrange the paragraph
to make the line shorter.  If you set ^|\emergencystretch|
\ctsref{\emergencystretch} to some nonzero
value, that may cure the problem by allowing \TeX\ to put more space between
words.  Another solution is to set |\tolerance| \ctsref{\tolerance}
to $10000$, but that's likely
to yield lines with far too much space in them.  Yet another solution is to
insert a discretionary hyphen
^^{discretionary hyphens//overfull boxes, correcting with}
in a critical word that \TeX\ didn't know how to hyphenate.
If all else fails, you might try rewording the paragraph.
A solution that is rarely satisfactory is increasing ^|\hfuzz| \ctsref\hfuzz,
allowing \TeX\ to construct lines that project beyond the right margin.

\li An underfull hbox that's a line of a paragraph indicates that the line was
too short and that \TeX\ couldn't rearrange the paragraph to
make the line longer.
\TeX\ will set such a line by stretching its interword spaces
beyond their normal
limits.  Two of the cures for overfull lines mentioned above
also apply to underfull lines:
inserting discretionary hyphens and rewording the paragraph.
Underfull lines
won't trouble you if you're using ragged right formatting, which you can get
with the ^|\raggedright| command \ctsref\raggedright.

\li The complaint:
\csdisplay
Underfull \vbox (badness 10000) has occurred
   while \output is active
|
indicates that \TeX\ didn't have enough material to fill up a page.  The
likely cause is that you've been using vboxes to keep material together
and \TeX\ has encountered a vbox near the bottom of a page that wouldn't fit on
that page.
It has put the vbox on the next page, but in doing so has left too much
empty space in the current page.
In this case you'll either have to insert some more space
elsewhere on the current page or break up the vbox into smaller parts.

Another possible cause of this complaint is having a long paragraph
that occupies an entire page without a break.
Since \TeX\ won't ordinarily vary the spacing between lines,
it may be unable to fill a gap at the bottom of the page
amounting to a fraction of the line spacing.
This can happen if |\vsize| \ctsref{\vsize}, the page length,
is not an even multiple of |\baselineskip| \ctsref{\baselineskip},
the space between consecutive
baselines.

Yet another cause of this complaint,
similar to the previous one, is setting ^|\parskip| \ctsref{\parskip},
the interparagraph glue, to a value that doesn't have any
stretch or shrink. You can fix these last two problems by increasing ^|\vfuzz|
\ctsref\vfuzz.

\li The complaint
\csdisplay
Overfull \vbox (296.30745pt too high) has occurred
   while \output is active
|
indicates that you constructed a vbox that was longer than the page.  You'll
just have to make it shorter.

\li 
\bix^^|\hbox//overfull box from|
\bix^^|\vbox//overfull box from|
The only cures for an overfull hbox or vbox that you've constructed
with the |\hbox| or |\vbox| commands
(\pp\xrefn\hbox, \xrefn\vbox)
are to take something out of the box, to insert some negative glue
with ^|\hss| or ^|\vss| (\xref\hss), or to increase the size
of the box.

\li If you encounter an underfull hbox or vbox that you've constructed 
with |\hbox| or |\vbox|,
you're usually best off to fill out the box with ^|\hfil| 
or ^|\vfil| (\xref\hfil).
\eix^^|\hbox//overfull box from|
\eix^^|\vbox//overfull box from|

\endulist
\eix^^{boxes//overfull}
\eix^^{boxes//underfull}
\eix^^{overfull boxes}
\eix^^{underfull boxes}


\section Recovering lost interword spaces

\null
^^{space//lost}
^^{words run together}
If you find that \TeX\ has run two words together, the likely cause is a
control sequence that's absorbed the spaces after it.  
Put a ^{control space} (|\!visiblespace|) after the control
sequence.


\section Avoiding unwanted interword spaces

\null
\bix^^{space//unwanted}\xrdef{unwantedspace}
If you get a space in your document where you don't want and don't
expect one, the most likely cause, in our experience,
is an end of line or a space following a brace.
^^{braces//space after}
(If you're doing fancy things with category codes,
^^{category codes//cause of unwanted spaces}
you've introduced lots of other likely causes.)
\TeX\ ordinarily translates an end-of-line into a space,
and it considers a space after a right or left brace to be significant.

If the unwanted space is caused by a space after a brace within
an input line, then remove that space.
If the unwanted space is caused by a brace at the end of an input line,
put a `|%|' immediately after the brace.
{\recat!ttidxref[%//to eliminate unwanted spaces]]
The `|%|' starts a comment, but this comment needn't
have any text.

A macro definition can also
introduce unwanted spaces if you haven't written it
carefully.  If you're getting unwanted spaces when you call a macro, check
its definition to be sure that you don't have an unintended space after a
brace and that you haven't ended a line of the definition immediately
after a brace.
People often end lines of macro definitions after braces in order to make
the definitions more readable.
To be safe, put a `|%|' after any brace that ends a line of a macro
definition.
It may not be needed, but it won't do any harm.\footnote{
Admittedly there are rare cases where you really do want an end of line
after a brace.}

When you're having trouble locating the source of an unwanted space,
try setting |\tracingcommands| \ctsref{\tracingcommands} to $2$.
You'll get a |{blank space}| command in the log file for each space
that \TeX\ sees.

It helps to know \TeX's rules for spaces:

\olist
\li Spaces are ignored at the beginnings of input lines.
\li Spaces at the ends of input lines are ignored under \emph{all}
circumstances, although the end of line itself is treated like a space.
(A completely blank line, however, generates a |\par| token.)
\li Multiple spaces are treated like a single space, but
only if they appear together in your input.
Thus a space following the arguments of a macro call is not combined
with a final space produced by the macro call.  Instead, you get two spaces.
\li Spaces are ignored after control words.
\li Spaces are in effect ignored after numbers, dimensions,
and the `|plus|' and `|minus|' in glue specifications.%
\footnote{Actually, \TeX\ ignores only a
single space in these places.
Since multiple spaces ordinarily reduce to a single space, however, the
effect is that of ignoring any number of spaces.}
\endolist
\noindent If you've changed the category code of the space or the end-of-line
character, all bets are off.
\eix^^{space//unwanted}


\section Avoiding excess space around a display

\null
^^{math display}
If you're getting too much space above a math display, it may be because
you've left a blank line in your input above the display.
The blank line starts a new paragraph and puts \TeX\ into 
vertical mode.  When
\TeX\ sees a `|$|' in vertical mode, it switches back to horizontal mode
and inserts the interparagraph glue (|\parskip|) followed by
the interline glue (|\baselineskip|).
Then, when it starts the display itself,
it inserts \emph{more} glue (either ^|\abovedisplayskip|
or ^|\abovedisplayshortskip|, depending on the length of the preceding
line).  
This last glue is the only glue that you want.
To avoid getting the interparagraph glue as well, don't leave a blank line above a
math display or otherwise
end a paragraph (with |\par|, say) just before a math display.

Similarly,
if you're getting too much space below a math display, it may be because
you've left a blank line in your input below the display.
Just remove it.


\section Avoiding excess space after a paragraph

If you get too much vertical space after a paragraph that was produced by a
macro, you may be getting the interparagraph glue produced by the macro,
an empty paragraph, and then more interparagraph glue.
You can get rid of the second paragraph skip by inserting:
\csdisplay
\vskip -\parskip
\vskip -\baselineskip
|
just after the macro call.
If you always get this problem with a certain macro, you can put these
lines at the end of the macro definition instead.
You may also be able to cure the problem by never leaving a blank line
after the macro call---if you want a blank line just to make your input
more readable, start it with a `|%|'.


\section Changing the paragraph shape

\null
\bix^^{paragraphs//shaping}
Several \TeX\ parameters---^|\hangindent|, ^|\leftskip|, etc.---%
affect the way that \TeX\ shapes paragraphs and breaks them into lines.
^^{line breaking}
These parameters are used indirectly in \plainTeX{} commands such as
^|\narrower| and ^|\hang|; you can also assign to them directly.
If you've used one of these commands (or changed one of these parameters),
but the command or parameter change
does not seem to be having any effect on a paragraph,
the problem may be that you've ended a group before you've ended the
paragraph.  For example:
\csdisplay
{\narrower She very soon came to an open field, with
a wood on the other side of it: it looked much darker
than the last wood, and Alice felt a little timid
about going into it.}
|
This paragraph won't be set narrower because the right brace at the end
terminates the |\narrower| group before \TeX\ has had a chance to
break the paragraph into lines.  Instead, put a |\par| before the
right brace; then you'll get the effect you want.
^^|\par//when changing paragraph shape|
\eix^^{paragraphs//shaping}


\section Putting paragraphs into a box

Suppose you have a few paragraphs of text that you want to put in a
particular place on the page.  The obvious way to do it is to enclose
the paragraphs in an hbox of an appropriate size, and then place the hbox
where you want it to be.  Alas, the obvious way doesn't work because
\TeX\ won't do line breaking in restricted horizontal mode.
^^{restricted mode//horizontal}
If you try it,
you'll get a misleading error message that suggests you're
missing the end of a group.
The way around this restriction is to write:
\csdisplay
\vbox{\hsize = !<dimen> !dots !<paragraphs> !dots}
|
where \<dimen> is the line length that you want for the paragraphs.
This is what you need to do, in particular, when you want to enclose some
paragraphs in a box (a box enclosed in ruled lines, not a \TeX\ box).

\section Drawing lines

\null
\bix^^{rules}
You can use the ^|\hrule| and ^|\vrule| commands (\xref\hrule)
to draw lines, i.e., rules.  You'll need to
know (a)~where you can use each command and
(b)~how \TeX\ determines the lengths of rules when
you haven't given the lengths explicitly.
\ulist
\li You can only use |\hrule| when \TeX\ is in a vertical mode and |\vrule|
when \TeX\ is in a horizontal mode.
This requirement means that 
you can't put a horizontal rule into an hbox or a vertical rule
into a vbox.
You can, however, construct a horizontal rule that looks vertical by
specifying all three dimensions and making it tall and skinny.
Similarly, you can construct a vertical rule that looks horizontal
by making it short and fat.

\li A horizontal rule inside a vbox
has the same width as does the vbox if you haven't
given the width of the rule explicitly.
Vertical rules
inside hboxes behave analogously.
If your rules are coming out too long or too short,
check the dimensions of the enclosing box.
\endulist

As an example, suppose we want to produce:
\display{%
\hbox{\vrule
   \vbox{\hrule\vskip 3pt
      \hbox{\hskip 3pt
          \vbox{\hsize .7in\raggedright
             \noindent Help! Let me out of here!}%
      \hskip 3pt}
   \vskip 3pt\hrule}%
\vrule}}
The following input will do it:
\csdisplay
\hbox{\vrule
   \vbox{\hrule \vskip 3pt
      \hbox{\hskip 3pt
          \vbox{\hsize = .7in \raggedright
             \noindent Help!! Let me out of here!!}%
      \hskip 3pt}%
   \vskip 3pt \hrule}%
\vrule}
|
We need to put the text into a vbox in order to get \TeX\ to process it
as a paragraph.
The four levels of boxing are really
necessary---if you doubt it, try to run this example with fewer levels.
\eix^^{rules}


\section Creating multiline headers or footers

\null
\xrdef{bighead}
\bix^^{headers//multiple-line}
\bix^^{footers//multiple-line}
You can use
the ^|\headline|  and ^|\footline| commands
(\xref\footline) to produce headers and footers, but they don't work properly
for headers and footers having more than one line.  However, you can get 
multiline headers and footers by redefining some of the subsidiary
macros in \TeX's output routine.

For a multiline header, you need to do three things:
\olist\compact
\li Redefine the ^|\makeheadline| macro that's called from \TeX's
output routine.
\li Increase ^|\voffset| by the amount of vertical space consumed by the
extra lines.
\li Decrease ^|\vsize| by the same amount.
\endolist
\noindent The following example shows how you might do this:
\csdisplay
\advance\voffset by 2\baselineskip
\advance\vsize by -2\baselineskip
\def\makeheadline{\vbox to 0pt{\vss\noindent
   Header line 1\hfil Page \folio\break
   Header line 2\hfil\break
   Header line 3\hfil}%
   \vskip\baselineskip}
|
You can usually follow the pattern of this definition quite closely, just
substituting your own header lines and choosing an appropriate multiple
of |\baselineskip| (one less than the number of lines in the header).

For a multiline footer, the method is similar:
\olist
\li Redefine the ^|\makefootline| macro that's called from \TeX's
output routine.
\li Decrease ^|\vsize| by the amount of vertical space consumed by the
extra lines.
\endolist
\noindent The following example shows how you might do this:
\csdisplay
\advance\vsize by -2\baselineskip
\def\makefootline{%
   \lineskip = 24pt
   \vbox{\raggedright\noindent
   Footer line 1\hfil\break
   Footer line 2\hfil\break
   Footer line 3\hfil}}
|
Again, you can usually follow the pattern of this definition quite closely.
The value of |\lineskip| determines the amount of space between the
baseline of the last line
of the main text on the page and the baseline of the first line of the footer.
\eix^^{headers//multiple-line}
\eix^^{footers//multiple-line}


\section Finding mismatched braces

\bix^^{braces//mismatched}
\xrdef{mismatched}
Most times when your \TeX\ input suffers from mismatched braces, you'll
get a diagnostic from \TeX\ fairly near the place where you actually
made the mistake.
But one of the most frustrating errors you can get from a \TeX\ run,
just before \TeX\ quits, is
the following:
\csdisplay
(\end occurred inside a group at level 1)
|
This indicates that there is an extra left brace 
or a missing right brace somewhere
in your document, but it gives you no hint at all
about where the problem might be.  So how can you find it?

A debugging trick we've found useful
is to insert the following line or its equivalent
at five or six places equally spaced within the document (and not within
a known group):
\csdisplay
}% a fake ending
|
Let's assume the problem is an extra left brace.
If the extra left brace
is, say, between the third and fourth fake ending,
you'll get error messages from the first three fake endings but not from
the fourth one.  The reason is that \TeX\ will ignore the first three
fake endings after complaining about them,
but the fourth fake ending will match
the extra left brace.
Thus you know that the extra left brace
is somewhere between the third and fourth
fake ending.
If the region of the error
is still too large for you to find it,
just remove the original set of fake endings and
repeat the process within that region.
If the problem is a missing right brace rather than an extra left brace,
you should be able to track it down once you've found its mate.

This method doesn't work under all circumstances.  In particular, it doesn't
work if your document consists of several really large groups.  But often
you can find some variation on this method that will lead you to that
elusive brace.

If all else fails, try shortening your input by removing 
the last half of the file
(after stashing away the original version first!) or inserting a
|\bye| command in the middle.
If the error persists, you know it's in the first half; if it goes away,
you know it's in the second half.
By repeating this process you'll eventually find the error.
\eix^^{braces//mismatched}


\section Setting dimensions

\null
\bix^^{\<dimen>}
The simplest way to set a dimension is to specify it directly, e.g.:
\csdisplay
\hsize = 6in
|
You can also specify a dimension in terms of other dimensions or as a mixture
of different units, but it's a little more work.  There are two ways
to construct a dimension as such a combination:
\olist\compact
\li You can add a dimension to a dimension parameter or to a dimension
register.
For example:
\csdisplay
\hsize = 6in \advance\hsize by 3pc % 6in + 3pc
|
\li You can indicate a dimension as a multiple of a dimension 
or glue parameter or register.
In this case, \TeX\ converts glue to a dimension by throwing away
the stretch and shrink.  For example:
\csdisplay
\parindent = .15\hsize
\advance\vsize by -2\parskip
|
\endolist
\eix^^{\<dimen>}


\section Creating composite fonts

\null
\bix^^{fonts//composite}
It's sometimes useful to create a ``composite font'',
named by a control sequence $\cal F$, in which all the characters
are taken from a font $f_1$ except for a few that are
borrowed from another font $f_2$.
You can then set text in the composite font by using $\cal F$ just as
you'd use any other font identifier.

You can create such a composite font by defining $\cal F$ as a macro.
In the definition of $\cal F$, you first select font $f_1$ and then
define control sequences that produce the borrowed characters,
set in $f_2$.
For example, suppose that you want to 
create a composite font |\britrm| which has all the characters of |cmr10|
except for the dollar sign, for which you want to borrow 
the pound sterling symbol
from font |cmti10|.
The pound sterling symbol in |cmti10| happens to be in the same font position
as the dollar sign in |cmr10|.
Here's how to do it:
\csdisplay
\def\britrm{%
   \tenrm % \tenrm names the cmr10 font
   \def\${{\tenit\char `\$}}% \tenit names the cmti10 font.
}
|
Now whenever you start the font named |\britrm|,
|\$| will produce a pound sterling symbol.

You can also get the same effect by changing the category codes
of the characters in question to make those characters active and then
providing a definition for the character.  For example:
\csdisplay
\catcode `* = \active
\def*{{\tentt \char `\*}}
|
In this case the asterisk will be taken from the |\tentt| font.
If you then type the input line:
\csdisplay
Debbie was the * of the show.
|
it will be set as:
{\font\tentt=cmtt10%
\catcode `* = \active
\def*{{\tentt \char `\*}}%
\display{Debbie was the * of the show.}}
\noindent

\eix^^{fonts//composite}


\section Reproducing text verbatim

\null
^^{verbatim text}\xrdef{verbatim}
Verbatim text is text that is reproduced in a typeset
document just as it appeared in the input.
The most common use of verbatim text is in typesetting
computer input,
including both computer programs and input to \TeX\ itself.
^^{computer programs, typesetting}
Computer input is not easy to produce verbatim for two reasons:
\olist\compact
\li Some characters (control symbols, escape characters, braces,
etc.) have
special meanings to \TeX.
\li Ends of line and multiple spaces are translated to single spaces.
\endolist
\noindent
In order to produce verbatim text, you have to cancel the special meanings
and disable the translation.  This is best done with macros.

To cancel the special meanings, you need to change the category codes
of those characters that have special meanings.
^^{category codes//for verbatim text}
The following macro illustrates how you might do it:
\csdisplay
\chardef \other = 12
\def\deactivate{%
   \catcode`\\ = \other   \catcode`\{ = \other
   \catcode`\} = \other   \catcode`\$ = \other
   \catcode`\& = \other   \catcode`\# = \other
   \catcode`\% = \other   \catcode`\~ = \other
   \catcode`\^ = \other   \catcode`\_ = \other
}
|
But beware! Once you've changed the category codes in this way, you've lost
the ability to use control sequences since there's no longer an escape
character.  You need some way of getting back to the normal mode of
operation.  We'll explain how to do that in a moment, after considering
the other problem: disabling the translation of spaces and ends of line.

\PlainTeX\ has two commands that together nearly solve the problem:
^|\obeyspaces| \ctsref{\obeyspaces} and ^|\obeylines| \ctsref{\obeylines}.
The two things that they don't do are to preserve spaces at the start of
a line
and to preserve blank lines.
For that you need stronger measures---which are provided by
the ^|\obeywhitespace| macro that we are about to define.

\TeX\ normally insists on collecting lines into paragraphs. One way to 
convince it to take line boundaries literally is to turn individual lines
into paragraphs.\footnote{%
Another way is to turn the end of line character into a |\break|
command and provide infinite glue at the end of each line.
^^|\break//end of line as|
}
You can do this by redefining the end of line character to produce
the |\par| control sequence.  The following three macro definitions
show how:

^^{whitespace, preserving}
\xrdef{\obeywhitespace}
\csdisplay
\def\makeactive#1{\catcode`#1 = \active \ignorespaces}
{% The group delimits the text over which ^^M is active.
   \makeactive\^^M %
   \gdef\obeywhitespace{%
   % Use \gdef so the definition survives the group.
      \makeactive\^^M %
      \let^^M = \newline %
      \aftergroup\removebox % Kill extra paragraph at end.
      \obeyspaces %
   }%
}
\def\newline{\par\indent}
\def\removebox{\setbox0=\lastbox}

|
A subtle point about the definition of |\obeywhitespace| is that
|^^M| must be made active both 
when |\obeywhitespace| is being \emph{defined} and when
it is being \emph{used}.

In order to be able to get back to normal operation after verbatim text,
you need to choose a character that
appears rarely if at all in the verbatim text.  This character serves as
a temporary escape character.
The vertical bar (|!||) is sometimes a good choice.
With this choice, the macros:
\csdisplay
\def\verbatim{\par\begingroup\deactivate\obeywhitespace
   \catcode `\!| = 0 % Make !| the new escape character.
}

\def\endverbatim{\endgroup\endpar}

\def\!|{!|}
|
will do the trick.  Within the verbatim text, you can use a double
vertical bar (|!|!||) to denote a single one, and you end the verbatim text
with |!|endverbatim|.

There are many variations on this technique:
\ulist
\li If a programming language has keywords, you can turn 
each keyword into a command that typesets
that keyword in boldface.  Each keyword in the input should then be
preceded by the temporary escape character.
\li If you have a character (again, let's assume it's the vertical bar)
that
\emph{never} appears in the verbatim text, you can make it active and
cause it to end the verbatim text.  The macro definitions then go like
this:
\csdisplay
{\catcode `\!| = \active
\gdef\verbatim{%
   \par\begingroup\deactivate\obeywhitespace
   \catcode `!| = \active
   \def !|{\endgroup\par}%
}}
|
\endulist

The ideas presented here provide only a simple approach to typesetting
computer programs.
Verbatim reproduction is often not as revealing or easy to read as a version
that uses typographical conventions to reflect the syntax
and even the semantics of the program.
If you'd like to pursue this subject further, we recommend the following book:

\smallskip{\narrower\noindent
Baecker, Ronald M., and Marcus, Aaron, {\sl Human Factors
and Typography for More Readable Programs}. Reading, Mass.:
Addison-Wesley, 1990.\par}


\section Using outer macros

\null
^^{forbidden control sequence}^^{incomplete conditional}
If \TeX\ complains about a ``forbidden control sequence'',
you've probably used an outer macro in a non-outer context
\seeconcept{outer}.
^^{macros//outer}
An outer macro is one whose definition is
preceded by ^|\outer|.
An outer macro can't be used in a macro argument, in a macro definition,
in the preamble of an alignment, or in conditional
text, i.e., text that will be expanded only when a conditional test
has a particular outcome.
^^{alignments//outer control sequence in}
Certain macros have been defined as outer because they aren't
intended to be used in these contexts and such a use is probably an error.
The only ways around this problem are to redefine the
macro or to move its use to an acceptable context.

Using an outer macro in an improper context can also cause
\TeX\ to complain about a runaway situation or an incomplete conditional.
The problem can be hard to diagnose because the error message
gives no hint as to what it is.
If you get such an error message, look around for a call on an outer macro.
You may not always know that a particular macro is outer, but the
command `^|\show||\a|' \ctsref{\show}
will show you the definition of |\a| and also tell you
if |\a| is outer.


\section Changing category codes

\null
\bix^^{category codes//changing}
Sometimes it's useful to make local changes to the category code of a
character in some part of your document.  For instance, you might be
typesetting a computer program
^^{computer programs, typesetting}
or something else that uses normally active
characters for special purposes.  You'd then want to deactivate those
characters so that \TeX\ will treat them as being like any other character.

If you make such a local change to the category code of a character,
you may sometimes be dismayed to find that
\TeX\ seems to be paying no attention whatsoever to your change.  
Two aspects of \TeX's behavior are likely causes:
\olist
\li \TeX\ determines the category code of an input character
^^{input characters}
and attaches it to the character when it reads in the character.
Let's say you read in a tilde (|~|) and later
change the category code of tildes,
but make the change
before \TeX's stomach has actually processed that \emph{particular\/} 
tilde \seeconcept{\anatomy}.
\TeX\ will still respond to that
tilde using the category code as it was before the change.  
This difficulty typically arises 
when the tilde is part of an argument to a macro
and the macro itself changes the category code of tilde.

\li When \TeX\ is matching a call of a macro to the definition of that macro,
it matches not just the characters in the parameter pattern but also their
category codes.
^^{macros//arguments of}
^^{macros//parameters of}
If the category code of a pattern character isn't equal to
the category code of the same character in the call, \TeX\ won't consider
the characters as matching.
This effect can produce mysterious results because it 
\emph{looks} as though the pattern should match.
For example, if you've defined a macro:
\csdisplay
\def\eurodate#1/#2/#3{#2.#1.#3}
|
then the slash character must have the same category code when you
call |\eurodate| as it had when you defined |\eurodate|.
\endolist
If the problem arises because the troublesome character is an argument to
a macro, then
the usual cure is to redefine the macro as a pair
of macros |\mstart| and |\mfinish|, where
|\mstart| is to be called before the argument text
and |\mfinish| is to be called after it.
|\mstart| then sets up the category codes and |\mfinish| undoes the change,
perhaps just by ending a group.
\eix^^{category codes//changing}


\section Making macro files more readable

\null
\bix^^{macros//making readable}
You can make a file of macros more readable by 
setting the category codes of space to $9$ (ignored
character) and ^|\endlinechar| \ctsref{\endlinechar} to $-1$
at the beginning of the file.
Then you can use spaces and ends of line freely in the
macro definitions without getting 
unwanted spaces when you call the macros.
The ignored characters won't generate spaces, but they'll still
act as terminators for control sequences.
If you really do want a space, you can still get it with the ^|\space|
command \ctsref\space.

Of course you'll need to restore the category codes of space and end of line
to their normal values ($10$ and $5$, respectively) at the end of the file.
You can do this either by enclosing the entire file in a group or by
restoring the values explicitly.  If you choose to enclose the file in a group,
then you should also set ^|\globaldefs| to $1$ so that
all the macro definitions will be global and thus visible outside of the group.

A miniature example of a macro file of this form is:

\csdisplay
\catcode `\  = 9  \endlinechar = -1

\def \makeblankbox #1 #2 {
   \hbox{\lower \dp0 \vbox{\hidehrule {#1} {#2}
   \kern -#1 % overlap rules
   \hbox to \wd0{\hidevrule {#1} {#2}%
      \raise \ht0  \vbox to #1{} % vrule height
      \lower \dp0 \vtop to #1{}  % vrule depth
      \hfil \hidevrule {#2} {#1} }
   \kern -#1 \hidehrule {#2} {#1} } }

\def\hidehrule #1 #2 {
   \kern -#1 \hrule height#1 depth#2 \kern -#2 }
\def\hidevrule #1 #2 {
   \kern -#1 {\dimen0 = #1 \advance \dimen0 by #2
   \vrule width \dimen0 } \kern -#2 }

\catcode `\  = 10  \endlinechar = `\^^M
|
\noindent
Without the changed category codes,
these macros would have to be written much more compactly, using fewer
spaces and more `|%|'s at the ends of lines.
\eix^^{macros//making readable}

\endchapter\byebye
